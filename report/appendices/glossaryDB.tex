\newglossaryentry{wireshark}{name={Wireshark},
description={Program used to analyze packet data sent between network nodes}}

\newglossaryentry{python}{name={Python},
description={A programming language}}

\newglossaryentry{mac}{name={Mac},
description={A brand of personal computers}}

\newglossaryentry{linux}{name={Linux},
description={An operating system}}

\newglossaryentry{c}{name={C},
description={A programming language}}

\newglossaryentry{clang}{name={clang},
description={A compiler front-end for different \Gls{c} programming languages}}

\newglossaryentry{gcc}{name={gcc},
description={THIS IS AN ACRONYM}}

\newglossaryentry{c++}{name={C++},
description={A programming language}}

\newglossaryentry{java}{name={Java},
description={A programming language}}

\newglossaryentry{lua}{name={Lua},
description={A programming language, often used for making \glspl{script}}}

\newglossaryentry{script}{name={script},
description = {A list of commands that are executed by a certain program, usually as an extension of the original functionality},
plural=scripts} 

\newglossaryentry{int}{name={int},
description={See \gls{integer}}}

\newglossaryentry{double}{name={double},
description={A data type in \Gls{c} that contains a double precision floating point number}}

\newglossaryentry{integer}{name={integer},
description={A data type in \Gls{c} that contains an integer},
plural=integers}

\newglossaryentry{char}{name={char},
description={A data type in \Gls{c} that contains a character or a small integer},
plural=chars}

\newglossaryentry{float}{name={float},
description={A data type in \Gls{c} that contains a floating point number},
plural=floats}

\newglossaryentry{pycparser}{name={pycparser},
description={A \Gls{c} \gls{parser} written in \Gls{python}}}

\newglossaryentry{ply}{name={ply},
description = {A \Gls{python} \gls{library} for creating \glspl{lexer} and \glspl{parser}}}

\newglossaryentry{parser}{name={parser},
description = {A program that receives input, checks it for correct syntax and builds a data structure representing the input},
plural=parsers}

\newglossaryentry{lexer}{name={lexer},
description = {A lexer is a program that converts a sequence of characters into a sequence of tokens},
plural=lexers}

\newglossaryentry{token}{name={token},
description = {A string of characters, categorized as a symbol according to a set of given rules},
plural=tokens}

\newglossaryentry{dissector}{name={dissector},
description = {Code that decodes \gls{packet} data and makes it readable by humans},
plural=dissectors}

\newglossaryentry{packet}{name={packet},
description = {Small block of data transmitted over a network},
plural=packets}

\newglossaryentry{utility}{name={utility},
description = {A small program that supports larger applications by doing certain tasks},
plural=utilities}

\newglossaryentry{library}{name={library},
description = {A collection of pre-written code for aiding programmers in the development process},
plural=libraries}

\newglossaryentry{ipc}{name={inter-process communication},
description = {The exchange of data that happens between \glspl{process}}}

\newglossaryentry{process}{name={process},
description = {A program running on a computer},
plural=processes}

\newglossaryentry{struct}{name={struct},
description = {Short for structure, it is a type that groups several \glspl{member} into a single object},
plural=structs}

\newglossaryentry{member}{name={member},
description = {A representation in \Gls{c} that contains data or behaviour of a \gls{struct}},
plural=members}

\newglossaryentry{binary}{name={binary},
description = {Two base arithmetic using the digits 0 and 1}}

\newglossaryentry{binary file}{name={binary file},
description = {A computer-readable file stored in \gls{binary} format}}

\newglossaryentry{nested struct}{name={nested struct},
description = {A \gls{struct} within another \gls{struct}},
plural={nested structs}}

\newglossaryentry{sparc}{name={SPARC},
description={A microprocessor architecture based on reduced instruction set computing}}

\newglossaryentry{version control system}{name={version control system},
description = {A system that ensures consistency of files when several people are collaborating on them},
plural={version control systems}}

\newglossaryentry{scrum}{name={Scrum},
description={A \gls{software development methodology}}}

\newglossaryentry{software development methodology}{name={software development methodology},
description={A framework used to structure, plan and control a development process},
plural={software development methodologies}}

\newglossaryentry{branch}{name={branch},
description={A feature of a \gls{version control system} that enables modifications of files in parallel, by duplicating the originating code},
plural=branches}

\newglossaryentry{distributed repository model}{name={distributed repository model},
description={A distributed approach to a \gls{version control system}},
plural={distributed repository models}}

\newglossaryentry{markup language}{name={markup language},
description={A language for specifying the processing, definition and presentation of text}}

\newglossaryentry{markup language, this is an acronym}{name={markup language, this is an acronym},
description={ffs}}

\newglossaryentry{capture file}{name={capture file},
description={A file containing the data that is captured from network or IPC traffic},
plural={capture files}}

\newglossaryentry{ascii}{name={ASCII},
description={A character encoding scheme}}

\newglossaryentry{character encoding scheme}{name={character encoding scheme},
description={A system that maps characters to something else, write more }}

\newglossaryentry{hexadecimal}{name={hexadecimal},
description={A number system where sixteen is the base}}

\newglossaryentry{hex dump}{name={hex dump},
description={A \gls{hexadecimal} view of computer data},
plural={hex dumps}}

\newglossaryentry{pcap-file}{name={pcap-file},
description={See \gls{capture file}},
plural={pcap-files}}

\newglossaryentry{protocol}{name={protocol},
description = {A system of rules for exchanging messages between machines},
plural=protocols}

\newglossaryentry{link-layer}{name={link-layer},
description = {The \gls{protocol} layer that is responsible for transferring data between two nodes}}

\newglossaryentry{repository}{name={repository},
description = {A central storage area where data is kept and maintained},
plural=repositories}

\newglossaryentry{Sun RPC}{name={Sun RPC},
description={The Unix equivalent of Remote Procedure Call}}

\newglossaryentry{corba}{name={corba},
description={A standard for enabling pieces of software written in different languages to work together as a single application}}

\newglossaryentry{asn1}{name={ASN.1},
description={This is an acronym}}

\newglossaryentry{makefile}{name={makefile},
description={A file that helps the make utility in the creation of executables from source code},
plural=makefiles}

\newglossaryentry{post-dissector}{name={post-dissector},
description = {A \gls{dissector} that is run after every other \gls{dissector} has been run},
plural={post-dissectors}}

\newglossaryentry{boolean}{name={boolean},
description={A data type that represents logical truth, it can have the values True or False},
plural=booleans}

\newglossaryentry{string}{name={string},
description={A string in \Gls{c} is a character string stored as an \gls{array} containing the characters},
plural=strings}

\newglossaryentry{C99}{name={C99},
description={A modern extension of \Gls{c}}}

\newglossaryentry{GCC-XML}{name={GCC-XML},
description={Acronym}}

\newglossaryentry{xml}{name={xml},
description={Angry}}

\newglossaryentry{Objective-C++}{name={Objective-C++},
description={A programming language}}

\newglossaryentry{Objective-C}{name={Objective-C},
description={A programming language}}

\newglossaryentry{AST}{name={abstract syntax tree},
description={A tree represention of a compiled program}}

\newglossaryentry{data serialization}{name={data serialization},
description={The process of converting a data structure to a storable format}}

\newglossaryentry{perl}{name={perl},
description={A programming language}}

\newglossaryentry{php}{name={php},
description={A scripting language}}

\newglossaryentry{Ruby}{name={Ruby},
description={A programming language}}

\newglossaryentry{Javascript}{name={Javascript},
description={A scripting language}}

\newglossaryentry{Eclipse}{name={Eclipse},
description={An application aiding computer programmers in software development}}

\newglossaryentry{header}{name={header},
description={define this},
plural=headers}

\newglossaryentry{enumerated named value}{name={enumerated named value},
description={A value of a \gls{enumerated type} in \Gls{c}},
plural={enumerated named values}}

\newglossaryentry{enumerated type}{name={enumerated type},
description={A data type in \Gls{c}  that contains a set of named values of the type}}


\newglossaryentry{enum}{name={enum},
description={See \gls{enumerated named value}},
plural=enums}


\newglossaryentry{union}{name={union},
description={define this},
plural=unions}

\newglossaryentry{array}{name={array},
description={A data type that can hold a collection of elements},
plural=arrays}

\newglossaryentry{preprocessor}{name={preprocessor},
description={Define this},
plural=preprocessors}

\newglossaryentry{include}{name={\#include},
description={A \Gls{c} directive that includes other \gls{header} files to the current file}}

\newglossaryentry{if}{name={\#if},
description={A \Gls{c} directive that executes a statement if a given expression holds true }}

\newglossaryentry{ifdef}{name={\#ifdef},
description={A \Gls{c} directive that checks if a given token has been defined}}

\newglossaryentry{define}{name={\#define},
description={A \Gls{c} directive that can be used to define a constant or create a macro}}

\newglossaryentry{trailers}{name={trailers},
description={define this}}

\newglossaryentry{bit string}{name={bit string},
description={An \gls{array} that stores individual bits},
plural={bit strings}}

\newglossaryentry{endian}{name={endian},
description={See \gls{endianness}},
plural=endians}

\newglossaryentry{endianness}{name={endianness},
description={Refers to the ordering of bytes in a word. A big-endian machine stores the most significant byte first, and a little-endian the least significant.}}

\newglossaryentry{batch mode}{name={batch mode},
description={define this}}

\newglossaryentry{batch processing}{name={batch processing},
description={See \gls{batch mode}}}

\newglossaryentry{x86}{name={x86},
description={The instruction set architecture used by Intel processors}}

\newglossaryentry{Windows}{name={Windows},
description={An operating system by Microsoft}}

\newglossaryentry{Solaris}{name={Solaris},
description={An operating system by Sun Microsystems}}

\newglossaryentry{x86-64}{name={x86-64},
description={An extension of the \gls{x86} instruction set that is compatible with 64-bit processors}}

\newglossaryentry{argparse}{name={argparse},
description={define this}}

\newglossaryentry{wildcard}{name={wildcard},
description={A character that can be used as a substitute for any other character}}

\newglossaryentry{x-86}{name={x-86},
description={The instruction set architecture used by Intel processors}}
