%=====================
\chapter{Requirements}
%=====================
%\begin{comment}
\label{chap:requirements}
This chapter describes an utility that creates Wireshark dissectors from C
header files. The dissectors must interpret binary representations of C
structs. In \autoref{sec:reqsoverview} we give a high level overview of the
utility and its requirements, \autoref{sec:usecases} provides use cases for the
utility, \autoref{sec:reqspriority} explains how we prioritizes the
requirements, \autoref{sec:reqscompl} explains how we estimate their
complexity, \autoref{sec:reqslist} lists all the functional and non-function
requirements, and \autoref{sec:prodbacklog} contains the complete product
backlog.

\section{Overview}
%-----------------
\label{sec:reqsoverview}
We are to create an utility that allows Wireshark to interpret the binary
representations of C-language structs. While C structs seldom are exchanged
across networks, they are sometimes used in inter-process communication. The
purpose of the utility described here is to provide Wireshark with the
capability of automatically dissecting the binary representation of a C struct,
as long as its definition is known.

The expected work flow for the utility is to read one or more C header files,
which contain struct definitions, and output Wireshark dissectors, implemented
in Lua scripts. A configuration file or source code annotations in the header
files may be used when additional configuration is required.

\subsection{List of requirements}
\autoref{tab:reqsoverview} is an overview of all the requirements. See
\autoref{tab:funcreq} and \autoref{tab:nonfuncreq} for more detailed
description of the requirements.

\begin{table}[H] \small \center
\caption{Requirements overview\label{tab:reqsoverview}}
\begin{tabular}{l l}
	\toprule ID & Description \\ \midrule
	FR1 & Read basic C struct definitions \\
	FR1-A & Support data types: int, float, char and boolean \\
	FR1-B & Support members of type enums \\
	FR1-C & Support members of type structs \\
	FR1-D & Support members of type unions \\
	FR1-E & Support member of type array \\
	\addlinespace
	FR2 & Generate Wireshark dissectors in Lua \\
	FR2-A & Display simple structs \\
	FR2-B & Support display of structs within structs \\
	FR2-C & Support Wireshark filter and search on attributes \\
	FR2-D & Recognize invalid values for a struct member \\
	\addlinespace
	FR3 & Support C preprocessor directives and macros \\
	FR3-A & Support \#include \\
	FR3-B & Support \#define and \#if \\
	FR3-C & Support \verb+WIN32+, \verb+_WIN64+, \verb+__sparc+ etc \\
	\addlinespace
	FR4 & Support user configuration \\
	FR4-A & Support valid ranges for struct members \\
	FR4-B & Support enumerated named values or a bit strings \\
	FR4-C & Custom handling of specific data types \\
	\addlinespace
	FR5 & Structs with headers and/or trailers \\
	\addlinespace
	FR6 & Handle input which size and endian depends on platform \\
	FR6-A & Flags specified for each platform \\
	FR6-B & Flags which signal the platform \\
	\addlinespace
	FR7 & Support parameters from command line \\
	FR7-A & Support parameters for c-header file \\
	FR7-B & Support for configuration file \\
	FR7-C & Support batch mode of c-header and configuration \\
	FR7-D & Don't regenerate dissectors \\
	\addlinespace
	NR1 & Run on latest Windows \& Solaris OS \\
	NR2 & Dissector run on Windows \& Solaris, Intel \& Sparc \\
	NR3 & User interface shall be command line \\
	NR4 & Sufficient documentation for generating Lua-scripts \\
	NR5 & Sufficient documentation for extending functionality \\
	NR6 & Code should follow PEP8 and PEP20 \\
	NR7 & Code should be documented by docstrings \\
	\bottomrule
\end{tabular}
\end{table}

\section{Use Cases}
%------------------
\label{sec:usecases}

\subsection{Actors}
An actor specifies a role played by an external person or thing that interact
with our utility. We have three types of actors to consider. First is the
primary actor which in our case is the user of our utility. He who feeds it a
C file to generate dissectors. A secondary actor is someone who configures our
utility to change the output of it. Finally we have an offstage actor which
does not use our utility himself, but uses the output dissectors in Wireshark.

We have defined two use case actors for our utility. The customer has
specified that the user is the most important actor.
\begin{description}
	\item[User] User of the generated Wireshark dissectors, offstage actor
	\item[Developer] User and configurer of utility, primary and secondary
		actor
\end{description}

\subsection{Use Case Diagrams}
TODO!! Desperately need help for this....

\section{Prioritization}
%-----------------------
\label{sec:reqspriority}
The team has, in cooperation with the customer, prioritized the requirements
in three categories:
\begin{inparaenum}[\itshape a\upshape)]
	\item High,
	\item Medium or
	\item Low.
\end{inparaenum} 

\begin{description}
	\item[High] Core functionality of the utility which must be implemented.
	\item[Medium] Requirements that will improve the value of the utility.
	\item[Low] Requirements that will not add much value to the utility.
\end{description}

\section{Complexity}
%-------------------
\label{sec:reqscompl}
The team has estimated the complexity for each requirement. We use the same
categories as for requirements priority:
\begin{inparaenum}[\itshape a\upshape)]
	\item High,
	\item Medium or
	\item Low.
\end{inparaenum} 

\begin{description}
	\item[High] Functionality which seems difficult and non-trivial to create.
	\item[Medium] Functionality that seems time consuming but straight forward.
	\item[Low] Requirements that are trivial to implement.
\end{description}
%\end{comment}

\section{List of requirements}
%-----------------------------
\label{sec:reqslist}
\autoref{tab:funcreq} lists the functional requirements, while
\autoref{tab:nonfuncreq} lists the non-functional requirements. Each
requirement have a priority (Pri) and a complexity (Cmp): High (H), 
Medium (M) or Low (L).

\begin{table}[ht] \footnotesize \center
\caption{Functional Requirements\label{tab:funcreq}}
\noindent\makebox[\textwidth]{%
\begin{tabularx}{1.2\textwidth}{l X c c}
	\toprule
	ID & Description & Pri. & Cmp. \\
	\midrule
	FR1 & The utility must be able to read basic C language struct definitions from C header files & H & \\
	FR1-A & The utility must support the following basic data types: int, float, char and boolean & H & L \\
	FR1-B & The utility must support members of type enums & H & L \\
	FR1-C & The utility must support members of type structs & H & M \\
	FR1-D & The utility must support members of type unions & M & M \\
	FR1-E & The utility must support member of type array & H & M \\
	\midrule
	FR2 & The utility must be able to generate lua-script for Wireshark dissectors for the binary representation of C struct & H & \\
	FR2-A & The dissector shall be able to display simple structs & H & L \\
	FR2-B & The dissector shall be able to support structs within structs & M & M \\
	FR2-C & The dissector must support Wiresharks built-in filter and search on attributes & H & L \\
	FR2-D & The dissector shall be able to recognize invalid values for a struct member & L & L \\
	\midrule
	FR3 & The utility must support C preprocessor directives and macros & H & \\
	FR3-A & The utility shall support \#include & H & L \\
	FR3-B & The utility shall support \#define and \#if & H & L \\
	FR3-C & The utility shall support \verb+WIN32+, \verb+_WIN32+, \verb+_WIN64+, \verb+__sparc__+, \verb+__sparc+ and \verb+sun+ & M & H \\
	\midrule
	FR4 & The utility must support user configuration & M & \\
	FR4-A & Configuration must support valid ranges for struct members & L & L \\
	FR4-B & Configuration must support integer members which represent enumerated named value or a bit string & M & L \\
	FR4-C & Configuration must support custom handling of specific data types. E.g. a 'time\_t' may be interpreted to contain a unixtime value, and be displayed as a date & L & M \\
	\midrule
	FR5 & A struct may have a header and/or trailer (other registered protocol). The configuration must support the use of integer members to indicate the number of other structs that will follow in the trailer & L & H \\
	\midrule
	FR6 & The dissectors must be able to handle binary input which size and endian depends on originating platform & M & \\
	FR6-A & Flags must be specified for each platform & M & M \\
	FR6-B & Flags within message headers should signal the platform & M & H \\
	\midrule
	FR7 & The utility shall support parameters from command line & H & \\
	FR7-A & Command line shall support parameters for c-header file & H & L \\
	FR7-B & Command line shall support for configuration file & H & L \\
	FR7-C & Command line shall support batch mode of c-header and configuration file & L & M \\
	FR7-D & When running batch mode, dissectors that already are generated, shall not be regenerated, if the source are not modified since last run & L & M \\
	\bottomrule
\end{tabularx}}
\end{table}

\begin{table}[ht] \footnotesize \center
\caption{Non-Functional Requirements\label{tab:nonfuncreq}}
\noindent\makebox[\textwidth]{%
\begin{tabularx}{1.2\textwidth}{l X c c}
	\toprule
	ID & Description & Pri. & Cmp. \\
	\midrule
	NR1 & The utility shall be able to run on latest Windows and Solaris operating system & M & L \\
	\addlinespace
	NR2 & The dissector shall be able to run on Windows x86, Windows x86-64, Solaris x86, Solaris x86-64 and Solaris SPARC & M & M \\
	\addlinespace
	NR3 & The utilities user interface shall be command line. No clicking! & H & L \\
	\addlinespace
	NR4 & The configuration shall have sufficient documentation to allow a person with no previous knowledge of the system to be able to use it to generate LUA-scripts after X hours of reading & M & M \\
	\addlinespace
	NR5 & The configuration should have sufficient documentation to allow a person, already proficient with the system, to understand the code well enough to be able to extend it’s functionality after Y hours of reading & M & M \\
	\addlinespace
	NR6 & The utility code should follow standard python coding convention as specified by PEP8, and try to follow python style guidelines defined by PEP20 & H & L \\
	\addlinespace
	NR7 & The utilities code should be documented by python docstrings which should explain the use of the code. Python modules, classes, functions and methods should have docstrings & M & L \\
	\bottomrule
\end{tabularx}}
\end{table}

\section{User Stories}

The developer in this context is the developers at Thales Norway AS. \newline
The administrator is this context is the users of the dissectors. \newline
\begin{tabular}{l p{10cm}}
\hline
FR01: 	&As a developer, I want that the utility can read basic C language struct definitions from C header files.\\
FR01A: 	&As a developer, I want to genereate a Lua-script from a c header files, with a struct that contains basic c data types(int, float, char, boolean)\\
FR01B: 	&As a developer, I want to genereate a Lua-script from a c header files, with a struct that contains enums\\
FR01C: 	&As a developer, I want to genereate a Lua-script from a c header files, with a struct that contains other structs\\
FR01D: 	&As a developer, I want to genereate a Lua-script from a c header files, with a struct that contains unions\\
FR01E: 	&As a developer, I want to genereate a Lua-script from a c header files, with a struct that contains array\\
\hline
FR02:		&As a developer, I want to generate lua-script for Wireshark dissectors for the binary representation of C struct.\\
FR02A: 	&As a administrator, I want that the dissector is able to display simple C structs.\\
FR02B: 	&As a administrator, I want that the dissector is able to display C struct, that contains other structs.\\
FR02C: 	&As a administrator, I want to use Wiresharks built-in filter and search on attributes, on the dissectors.\\
FR02D:	&As a administrator, I want see when a there is a invalid value for a struct member\\
\hline
FR03:	&As a developer, I want that the utility supports C preprocessor directives and macros\\
FR03A:	&As a developer, I want support for \#include in the utility\\
FR03B:	&As a developer, I want support for \#define and \#if in the utility\\
FR03C:	&As a developer, I want that the utility supports WIN32, \_WIN32, \_WIN64, \_\_sparc\_\_, \_\_sparc and sun macros.\\
\hline
FR04: 	&As a developer, I want that the utility supports user configuration\\
FR04A: 	&As a developer, I want to specifiy allowed ranges for the value of a struct member\\
FR04B:	&As a developer, I want support for integer member which represent enumreated named value or bit string, in configuration\\
FR04C:	&As a developer, I want support for custom handling of specified data types(e.g. time\_t)\\
\hline
FR05:		&As a developer, I want support in configuration to handle structs with header and/or trailer\\
\hline
FR06:		&As a developer, I want dissectors that are able to handle binary input which size and denian depends on originating platform.\\
FR06A:	&As a developer, I want specified flags for each platform.\\
FR06B:	&As a developer, I want that flags within message headers should signal the platform\\
\hline
FR07:		&As a developer, I want to specify parameters from command-line\\
FR07A:	&As a developer, I want to specify parameters for c-header file from command-line\\
FR07B:	&As a developer, I want to specify parameters for configuration file from command-line\\
FR07C:	&As a developer, I want start a batch mode of c-header and configuration file from command-line\\
FR07D:	&As a developer, I do not want to regenerate dissectors that not are modified since last run, when running batch mode\\
\hline
\end{tabular}