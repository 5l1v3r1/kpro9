%=================
\chapter{Introduction}
%=================

This chapter will give a brief introduction to CSjark.

The first section will briefly explain Wireshark, dissectors and how dissectors are used in Wireshark.

The second section will describe how Lua code is built by our utility. 
The connection between Wireshark and the Lua structs protocol will also be explained.

The third, and last section gives an overview of the entire process of debugging a C struct, from the parsing of the struct definition, through the
generation of the dissector and to displaying it in dissected form in Wireshark.

\section{Wireshark and dissectors}
This section gives a brief introduction to Wireshark and dissectors.
The first part describes what Wireshark is, and what it can be used for.
The second part explains exactly what a dissector is, and how a dissector can be used to extend Wireshark.

\subsection{Wireshark}
Wireshark is a program used to analyze network traffic. A common usage scenario is when a person wants to troubleshoot network problems or
look at the internal workings of a network protocol. An important feature of Wireshark is the ability to look at a live stream of packets sent through the network.
It is also possible to filter and search on given packet attributes, which facilitates the debugging process.

\subsection{Dissectors}
Wireshark can be extended with a dissector. In short, a dissector detects the protocol type of a packet and tries to obtain as much information
from the packet as possible. This information is then displayed in Wireshark, in addition to the name of the protocol.


 A dissector knows the structure of the message and can represent it beautifully.

\section{Automated generation of Lua dissectors}
This section contains information on how Lua dissectors are generated.
The first part explains in general what a Lua dissector does to dissect packets.
The second part specifically explains how CSjark builds a Lua dissector.

\subsection{How Lua dissectors are built}

Proto - a new protocol in Wireshark. 'Protocols' can dissect a protocol.
Proto.dissector - the function doing the dissecting
ProtoField - a Protocol field, used when adding items to the dissection tree.
Proto.fields - contains the different ProtoFields defined
DissectorTable - a table with the subdissectors of a protocol

\subsection{The Lua structs protocol}

\section{Debugging C structs}

CSjark 

\subsection{Debugging C structs before CSjark}

\subsection{Debugging C structs now}

