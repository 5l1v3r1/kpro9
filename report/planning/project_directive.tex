%==========================
\chapter{Project Directive}
%==========================
This chapter will briefly introduce the project, its background and purpose.


%------------------------
\section{Project Mandate}
%------------------------
The purpose of this project is to develop an \gls{utility} that automatically creates \Gls{lua}-\glspl{dissector} for \Gls{wireshark}, from \Gls{c}-\gls{header} files. This report presents the team’s process from the initial requirement specification to the finished product. 

The title of the project is "\Gls{wireshark} - Automated generation of \gls{protocol} \glspl{dissector}", it was given to us by the customer and describes exactly what we are planning to accomplish.The name chosen for the \gls{utility} is "CSjark". Sjark is the Norwegian name for an iconic type of fishing boat, most commonly used in Northern Norway. The reason why the team picked this name was because of the way the \gls{utility} "fishes" for \Gls{c}-\glspl{struct} in \gls{header} files. The \gls{utility} then creates \glspl{dissector} for these \glspl{struct} so that \Gls{wireshark} can display the \gls{struct} information properly. This reminded the team of what fishermen do to prepare the fish for the market. The word Sjark is also pronounced in a similar way to "shark", which makes our \gls{utility}-name a play on words when comparing it to "\Gls{wireshark}", the program our \gls{utility} is supposed to work with.


%-------------------
\section{The Client}
%-------------------
The client for this project is
Thales Norway AS\footnote{\url{http://www.thales.no/}}. Thales is an
international electronics and systems group, which focuses on defence,
aerospace and security markets worldwide. The Norwegian branch primarily
supplies military communication systems, used by the Norwegian Armed Forces
and other \Gls{nato} countries. Thales Norway AS consists of 170 highly skilled
employees, which offers a wide range of technical competence.


%--------------------------------
\section{Involved parties}
%--------------------------------
The following parties are involved in this project.

The client, described in the section above, is represented by Christian Tellefsen and Stig Bjørlykke. See \autoref{tab:plan:customer} for their contact information.
The project team consists of seven computer engineering students from \Gls{ntnu}, as described in \autoref{tab:plan:devs}.
For feedback and help during the project period our team has been assigned a main supervisor, Daniela Soares Cruzes.
She will be assisted by Maria Carolina Passos. Their contact information can be found in \autoref{tab:plan:advisors}.

%----------------------------
\section{Project Background}
%---------------------------
Thales currently uses \Gls{wireshark} to analyze traffic data between different network nodes, for example, \Gls{ip} \glspl{packet} sent between a client and a server.
They want us to extend the functionality of \Gls{wireshark}, so that they could use it to monitor internal data between \glspl{process}. The extended functionality will be automatic generation of \Gls{lua}-\glspl{dissector} from \Gls{c}-\gls{header} files.

Before, when Thales wanted to debug with \Gls{wireshark}, they had to write the \glspl{dissector} manually. Their hope is therefore that our tool can save them valuable time.

%--------------------------
\section{Project Objective}
%--------------------------
The objective from the customer is to design a \gls{utility} that will be able to generate \Gls{lua} code for dissecting the \gls{binary} representation of \Gls{c}/\Gls{c++} \glspl{struct}, allowing \Gls{wireshark} to show, filter, and search through the data.
 The \gls{utility} needs to support a flexible configuration, as this will make it useable for complex and specific \gls{header} files. 
The code and configuration should be well documented, making it easy for Thales to use and extend the tool as they see fit.

The objective from \Gls{ntnu}'s point of view is that we acquire practical experience in executing all phases of a bigger \Gls{it}-project and learn
how to work in a team.

Our team's goals for this project are to attain experience in working in a real development project, and to create a solution that the customer
is satisfied with.

%-----------------
\section{Duration}
%-----------------
Calculations done by the course staff suggests that each student should conduct 325 person-hours distributed over 13 weeks for the project. Our group, consisting of seven students, will have a total of 2275 person-hours to spend.\\
\begin {itemize}
	\item Project start: August 30th
	\item Project end (Presentation): November 24th
\end{itemize}

