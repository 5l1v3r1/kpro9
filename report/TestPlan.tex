\documentclass[a4paper, 11pt]{report}
\usepackage[T1]{fontenc}
\usepackage[utf8]{inputenc}
\usepackage[english]{babel}
\usepackage{graphicx} % support graphics
\usepackage{hyperref} % links in the document
\usepackage{float} % position of figures
\usepackage{paralist} % inline lists

%\setcounter{tocdepth}{1} % Depth of table of contents

% Configure links in pdfs
\hypersetup{
    bookmarksopen=false, % Hide bookmarks menu
    colorlinks=true, % Don't wrap links in colored boxes
}

\title{Wireshark:\\ Automated generation of protocol dissectors\\
		Project Requirements}
\author{by Erik Bergersen, Sondre Johan Mannsverk,\\ Terje Snarby,
		Even Wiik Thomassen, Lars Solvoll Tønder,\\ Sigurd Wien
		and Jaroslav Fibichr}
\date{\today}

\begin{document}

\chapter{Test Plan}

This chapter presents the test plan for our solution. The test plan is based on  the standards set by the IEEE829-1998 standard for software testing, but with a few changes to better fit with our project. The purpose of this plan is to have a structured way of preforming tests, as well as providing the developers with a list of specific component-behaviours, based on the functional requirements. This will assist the developers working on the different components by giving them a list of objectives that should be solved during development of the different components. The tests will also assure us that 

\section{Templates for testing}

Here we will present the different templates we'll be using for testing purposes\\

\begin{table}[H]
\begin{tabular}{| l | l |}
Portion & Description\\
\hline\hline
Test identifier & Test ID\\
Tester & Team member responsible for the test\\
Prerequisites & Conditions that needs to be fulfilled before starting the test\\
Feature & Feature to test\\
Execution & Steps to be executed in the test\\
\hline
\end{tabular}
\caption{Test case template}
\end{table}

\begin{table}[H]
\begin{tabular}{| l | l |}
Portion & Description\\
\hline\hline
Test identifier & Test ID\\
Tester & Team member responsible for the test\\
Date & The date the testing took place\\
Result & Success \\
\end{tabular}
\caption{Test report template}
\end{table}

\end{document}
