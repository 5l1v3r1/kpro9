
%==============
\chapter{Test Cases}
%==============
\label{sec:testcases}

%-----------------------
\section{Sprint 1 Tests}
%-----------------------
Test cases for sprint 1 is \autoref{tab:TID01}-\ref{tab:TID07}. 

\begin{table}[ht] \footnotesize \center
\caption{Test case TID01}\label{tab:TID01}
\noindent\makebox[\textwidth]{%
\begin{tabularx}{1.2\textwidth}{l X}
	\toprule
	Header & Description \\
	\midrule
	Description & Supporting parameters for \Gls{c}-\gls{header} file \\
	Tester & Lars Solvoll Tønder \\
	Prerequisites & The utility must have been installed on the system and there needs to exist a header file associated with this test \\
	Feature & Test that we are able to feed the solution with a \Gls{c}-\gls{header} file and have it get dissected \\
	\midrule
	\multirow{3}{*}{Execution} & 1. Start the program \\
	& 2. Write the name of the \Gls{c}-\gls{header} file in the command line \\
	& 3. Read the output given by the program \\
	\midrule
	\multirow{2}{*}{Expected result} & 1. The program should start up without any errors and present the user with a command line interface \\
	& 3. The user should be presented with some text expressing the success of the \Gls{lua}-file generation \\
	\bottomrule
\end{tabularx}}
\end{table}

\begin{table}[ht] \footnotesize \center
\caption{Test case TID02}\label{tab:TID02}
\noindent\makebox[\textwidth]{%
\begin{tabularx}{1.2\textwidth}{l X}
	\toprule
	Header & Description \\
	\midrule
	Description & Supporting basic data types \\
	Tester & Lars Solvoll Tønder \\
	Prerequisites & The utility must have been installed on the system and there needs to exist a header file associated with this test \\
	Feature & Test that our \gls{utility} will be able to make a \glspl{dissector} for \Gls{c}-\gls{header} files including the following basic data types: \gls{int}, \gls{float}, \gls{char} and \gls{boolean} \\
	\midrule
	\multirow{2}{*}{Execution} & 1. Write the name of a \Gls{c}-\gls{header} file which includes the aforementioned basic data types \\
	& 2. Read the output given by the program \\ 
	\midrule
	Expected result & 2. The program should provide the user with some text expressing the success of the \Gls{lua}-file generation \\
	\bottomrule
\end{tabularx}}
\end{table}

\begin{table}[ht] \footnotesize \center
\caption{Test case TID03}\label{tab:TID03}
\noindent\makebox[\textwidth]{%
\begin{tabularx}{1.2\textwidth}{l X}
	\toprule
	Header & Description \\
	\midrule
	Description & Displaying simple \glspl{struct} \\
	Tester & Lars Solvoll Tønder \\
	Prerequisites & The \gls{utility} has already made a \gls{dissector} \\
	Feature & Test that our \gls{utility} is able to generate \glspl{dissector} that displays simple \glspl{struct} \\
	\midrule
	\multirow{3}{*}{Execution} & 1. Open \Gls{wireshark} and run the \gls{dissector} \\
	& 2. Run the \gls{dissector} on some captured data of simple \glspl{struct} \\
	& 3. Read the output \\
	\midrule
	\multirow{2}{*}{Expected result} & 1. \Gls{wireshark} should be able to load the \gls{dissector} without any errors \\
	& 3. \Gls{wireshark} should display the data inside the \glspl{struct} sent in the capture data \\
	\bottomrule
\end{tabularx}}
\end{table}

\begin{table}[ht] \footnotesize \center
\caption{Test case TID04}\label{tab:TID04}
\noindent\makebox[\textwidth]{%
\begin{tabularx}{1.2\textwidth}{l X}
	\toprule
	Header & Description \\
	\midrule
	Description & Supporting \gls{include} \\
	Tester & Lars Solvoll Tønder \\
	Prerequisites & The \gls{utility} has to be up and running \\
	Feature & Test that our \gls{utility} supports \Gls{c}-\gls{header} files with the \gls{include} directive \\
	\midrule
	\multirow{2}{*}{Execution} & 1. Write the name of a \Gls{c}-\gls{header} file with an \gls{include} directive \\
	& 2. Read the output \\
	\midrule
	Expected result & 2. The program should provide the user with some text expressing the success of the \Gls{lua}-file generation \\
	\bottomrule
\end{tabularx}}
\end{table}

\begin{table}[ht] \footnotesize \center
\caption{Test case TID05}\label{tab:TID05}
\noindent\makebox[\textwidth]{%
\begin{tabularx}{1.2\textwidth}{l X}
	\toprule
	Header & Description \\
	\midrule
	Description & Supporting \gls{define} and \gls{if} \\
	Tester & Lars Solvoll Tønder \\
	Prerequisites & The \gls{utility} has to be up and running \\
	Feature & Test that our \gls{utility} supports \Gls{c}-\gls{header} files with \gls{define} and \gls{if} directives \\
	\midrule
	\multirow{2}{*}{Execution} & 1. Write the name of a \Gls{c}-\gls{header} file with a \gls{define} and \gls{if} directives \\
	& 2. Read the output \\
	\midrule
	Expected result & 2. The program should provide the user with some text expressing the success of the \Gls{lua}-file generation \\
	\bottomrule
\end{tabularx}}
\end{table}

\begin{table}[ht] \footnotesize \center
\caption{Test case TID06}\label{tab:TID06}
\noindent\makebox[\textwidth]{%
\begin{tabularx}{1.2\textwidth}{l X}
	\toprule
	Header & Description \\
	\midrule
	Description & Supporting configuration files \\
	Tester & Lars Solvoll Tønder \\
	Prerequisites & The \gls{utility} has to be up and running \\
	Feature & Test that our \gls{utility} supports reading data from a configuration file \\
	\midrule
	\multirow{2}{*}{Execution} & 1. Feed the \gls{utility} with the name of a config file \\
	& 2. Read the output \\
	\midrule
	Expected result & 2. The program should provide the user with some text expressing either the success of reading the file, or the failure of reading the file due to any errors in the config-file itself\\
	\bottomrule
\end{tabularx}}
\end{table}

\begin{table}[ht] \footnotesize \center
\caption{Test case TID07}\label{tab:TID07}
\noindent\makebox[\textwidth]{%
\begin{tabularx}{1.2\textwidth}{l X}
	\toprule
	Header & Description \\
	\midrule
	Description & Recognizing invalid values \\
	Tester & Lars Solvoll Tønder \\
	Prerequisites & The \gls{utility} must have already loaded a configuration file and generated a \gls{dissector} for a given \gls{header} file\\
	Feature & Test that our \gls{utility} recognizes invalid values for \gls{struct} \glspl{member} specified in the config file. \\
	\midrule
	\multirow{3}{*}{Execution} & 1. Feed the \gls{utility} with the name of a \gls{header} and a config-file, where the config file sets restrictions on the \glspl{member} of the \gls{header} file. \\
	& 2. Use the resulting \gls{dissector} on some capture data for \Gls{wireshark} that contains \gls{struct} \glspl{member} with invalid values. \\
	& 3. Read the displayed data generated by \Gls{wireshark} \\
	\midrule
	Expected result & 3. \Gls{wireshark} should display the \gls{struct} \glspl{member} with invalid values as invalid \\
	\bottomrule
\end{tabularx}}
\end{table}


%-----------------------
\section{Sprint 2 Tests}
%-----------------------
Test cases for sprint 2 is \autoref{tab:TID08}-\ref{tab:TID14}. 

\begin{table}[ht] \footnotesize \center
\caption{Test case TID08}\label{tab:TID08}
\noindent\makebox[\textwidth]{%
\begin{tabularx}{1.2\textwidth}{l X}
	\toprule
	Header & Description \\
	\midrule
	Description & Supporting \glspl{member} of type \gls{enum} \\
	Tester & Lars Solvoll Tønder \\
	Prerequisites & The utility must have been installed on the system and there needs to exist a header file associated with this test \\
	Feature & Test that the \gls{utility} is able to support \Gls{c}-\gls{header} files with \glspl{enum} \\
	\midrule
	\multirow{5}{*}{Execution} & 1. Feed the \gls{utility} with the name of a \Gls{c}-\gls{header} file which includes a \gls{struct} using \glspl{enum} and its configuration file \\
	& 2. Read the output \\
	& 3. Move the resulting \gls{dissector} into the plugins folder in the personal configuration folder of \Gls{wireshark}\\
	& 4. Open the \gls{pcap-file} for this test with \Gls{wireshark}\\
	& 5. Look at the different \glspl{packet} and \gls{struct} \glspl{member} that are displayed in \Gls{wireshark}\\
	\midrule
	\multirow{2}{*}{Expected result } & 2. The program should provide the user with some text expressing the success of the \Gls{lua}-file generation \\ 
	\addlinespace
	& 5 The different \glspl{packet} should be displayed as having \glspl{struct} with \glspl{enum}, showing the value of the \gls{enum} by its name and value\\
	\bottomrule
\end{tabularx}}
\end{table}

\begin{table}[ht] \footnotesize \center
\caption{Test case TID09}\label{tab:TID09}
\noindent\makebox[\textwidth]{%
\begin{tabularx}{1.2\textwidth}{l X}
	\toprule
	Header & Description \\
	\midrule
	Description & Supporting \glspl{member} of type \gls{array} \\
	Tester & Lars Solvoll Tønder \\
	Prerequisites & The \gls{utility} has to be up and running \\
	Feature & Test that the \gls{utility} iis able to support \Gls{c}-\gls{header} files with \glspl{array} \\
	\midrule
	\multirow{5}{*}{Execution} & 1. Feed the \gls{utility} with the name of a \Gls{c}-\gls{header} file which includes a \gls{struct} using \glspl{array} \\
	& 2. Read the output \\
	& 3. Move the resulting \gls{dissector} into the plugins folder in the personal configuration folder of \Gls{wireshark}\\
	& 4. Open the \gls{pcap-file} for this test with \Gls{wireshark}\\
	& 5. Look at the different \glspl{packet} and \gls{struct} \glspl{member} that are displayed in \Gls{wireshark}\\
	\midrule
	\multirow{5}{*}{Expected result} & 2. The program should provide the user with some text expressing the success of the \Gls{lua}-file generation \\
	& 5 The different \glspl{packet} should be displayed as having \glspl{struct} with \glspl{array}, showing the values of the cells in the \gls{array}. Multidimensional \glspl{array} should also be displayed as being \glspl{array} with subtrees of other \glspl{array}. This should be indicated by the multidimensional \glspl{array} having a clickable "+" box to the left which when pressed shows the values of the cells of the inner \glspl{array} \\
	\bottomrule
\end{tabularx}}
\end{table}

\begin{table}[ht] \footnotesize \center
\caption{Test case TID10}\label{tab:TID10}
\noindent\makebox[\textwidth]{%
\begin{tabularx}{1.2\textwidth}{l X}
	\toprule
	Header & Description \\
	\midrule
	Description & Supporting the display of \glspl{struct} within \glspl{struct} \\
	Tester & Erik Bergersen \\
	Prerequisites & The \gls{utility} has to be up and running \\
	Feature & Test that the \gls{utility} iis able to support \Gls{c}-\gls{header} files with \glspl{struct} that have other \gls{struct} \glspl{member}  and display it properly in \Gls{wireshark}\\
	\midrule
	\multirow{5}{*}{Execution} & 1. Feed the \gls{utility} with the name of a \Gls{c}-\gls{header} file which includes a \gls{struct} with another \gls{struct} \gls{member} \\
	& 2. Read the output \\
	& 3. Move the resulting \gls{dissector} into the plugins folder in the personal configuration folder of \Gls{wireshark}\\
	& 4. Open the \gls{pcap-file} for this test with \Gls{wireshark}\\
	& 5. Look at the different \gls{struct} \glspl{member} that are displayed in \Gls{wireshark}\\
	\midrule
	\multirow{2}{*}{Expected result} & 2. The program should provide the user with some text expressing the success of the \Gls{lua}-file generation\\
	& 5. The \gls{struct} \glspl{member} within \glspl{struct} should be displayed as subtrees, indicated by having a clickable "+" box to the left whch when clicked, displayes the contents of the given \gls{struct}\\
	\bottomrule
\end{tabularx}}
\end{table}

\begin{table}[ht] \footnotesize \center
\caption{Test case TID11}\label{tab:TID11}
\noindent\makebox[\textwidth]{%
\begin{tabularx}{1.2\textwidth}{l X}
	\toprule
	Header & Description \\
	\midrule
	Description & Supporting enumerated named values \\
	Tester & Erik Bergersen \\
	Prerequisites & The \gls{utility} has to be up and running \\
	Feature & Test that the \gls{utility} iis able to support \Gls{c}-\gls{header} files which uses integers as \glspl{enum} without declaring them as \glspl{enum} \\
	\midrule
	\multirow{5}{*}{Execution} & 1. Feed the \gls{utility} with the name of a \Gls{c}-\gls{header} file which includes enumerated named values \\
	& 2. Read the output \\
	& 3. Move the resulting \gls{dissector} into the plugins folder in the personal configuration folder of \Gls{wireshark}\\
	& 4. Open the \gls{pcap-file} for this test with \Gls{wireshark}\\
	& 5. Look at the different \gls{struct} \glspl{member} that are displayed in \Gls{wireshark}\\
	\midrule
	\multirow{2}{*}{Expected result}  & 2. The program should provide the user with some text expressing the success of the \Gls{lua}-file generation \\
	\addlinespace
	& 5. The different \glspl{packet} should be displayed as containing enumerated named values expressed by their name and not value\\
	\bottomrule
\end{tabularx}}
\end{table}

\begin{table}[ht] \footnotesize \center
\caption{Test case TID12}\label{tab:TID12}
\noindent\makebox[\textwidth]{%
\begin{tabularx}{1.2\textwidth}{l X}
	\toprule
	Header & Description \\
	\midrule
	Description & Supporting \glspl{bit string} \\
	Tester & Erik Bergersen \\
	Prerequisites & The \gls{utility} has to be up and running \\
	Feature & Test that the \gls{utility} iis able to support \Gls{c}-\gls{header} files with \glspl{bit string} \\
	\midrule
	\multirow{5}{*}{Execution} & 1. Feed the \gls{utility} with the name of a \Gls{c}-\gls{header} file with a \gls{struct} using \glspl{bit string} \\
	& 2. Read the output \\
	& 3. Move the resulting \gls{dissector} into the plugins folder in the personal configuration folder of \Gls{wireshark}\\
	& 4. Open the \gls{pcap-file} for this test with \Gls{wireshark}\\
	& 5. Look at the different \gls{struct} \glspl{member} that are displayed in \Gls{wireshark}\\
	\midrule
	\multirow{2}{*}{Expected result} & 2. The program should provide the user with some text expressing the success of the \Gls{lua}-file generation \\
	& 5. The \glspl{packet} should be displayed as containing \glspl{bit string}. \Glspl{bit string} should be displayed as subtrees indicated by a clickable "+" box to the left which shows the values of the different bits in the \gls{bit string} when pressed\\
	\bottomrule
\end{tabularx}}
\end{table}

\begin{table}[ht] \footnotesize \center
\caption{Test case TID13}\label{tab:TID13}
\noindent\makebox[\textwidth]{%
\begin{tabularx}{1.2\textwidth}{l X}
	\toprule
	Header & Description \\
	\midrule
	Description & Supporting \glspl{struct} with various \gls{trailers} \\
	Tester & Erik Bergersen \\
	Prerequisites & The \gls{utility} has to be up and running \\
	Feature & Test that the \gls{utility} iis able to support \Gls{c}-\gls{header} files with \gls{trailers} \\
	\midrule
	\multirow{5}{*}{Execution} & 1. Feed the \gls{utility} with the name of a \Gls{c}-\gls{header} file with a \gls{struct} that has a trailer \\
	& 2. Read the output \\
	& 3. Move the resulting \gls{dissector} into the plugins folder in the personal configuration folder of \Gls{wireshark}\\
	& 4. Open the \gls{pcap-file} for this test with \Gls{wireshark}\\
	& 5. Look at the different \gls{struct} \glspl{member} that are displayed in \Gls{wireshark}\\
	\midrule
	\multirow{2}{*}{Expected result}  & 2. The program should provide the user with some text expressing the success of the \Gls{lua}-file generation \\
	& 5. The \glspl{packet} should be display the data inside the \gls{struct}, and the data from the trailer should be displayed below\\
	\bottomrule
\end{tabularx}}
\end{table}

\begin{table}[ht] \footnotesize \center
\caption{Test case TID14}\label{tab:TID14}
\noindent\makebox[\textwidth]{%
\begin{tabularx}{1.2\textwidth}{l X}
	\toprule
	Header & Description \\
	\midrule
	Description & Sprint 2 functionality test \\
	Tester & Lars Solvoll Tønder \\
	Prerequisites & The \gls{utility} has to have been installed on the system as well as the attest testing framework \\
	Feature & Checking that the \gls{utility} is able to create a valid \gls{dissector} from \gls{header} files with all of the data types that were to be supported for sprint 2, including: \gls{trailers}, \glspl{bit string}, enumerated named values, \glspl{struct} within \glspl{struct} and \glspl{array} \\
	\midrule
	\multirow{3}{*}{Execution} & 1. Navigate to the folder where CSjark is installed through the terminal or command line \\
	& 2.  type "python -m attest" into the terminal or command line and then press enter  \\
	& 3. Read the output\\
	\midrule
	Expected result  & 3. The user should be presented with some text expressing the failure of 0 assertions  \\
	\bottomrule
\end{tabularx}}
\end{table}

%------------------------------
\section{Sprint 3 Tests}
%-----------------------------
Test cases for sprint 3 is \autoref{tab:TID15}-\ref{tab:TID23}. 

\begin{table}[ht] \footnotesize \center
\caption{Test case TID15}\label{tab:TID15}
\noindent\makebox[\textwidth]{%
\begin{tabularx}{1.2\textwidth}{l X}
	\toprule
	Header & Description \\
	\midrule
	Description & Support \gls{batch mode} of \Gls{c} \gls{header} and configuration files \\
	Tester & Lars Solvoll Tønder \\
	Prerequisites & The \gls{utility} has have been installed on the system, there also needs to exist a header and configuration file for this test \\
	Feature &  Test that the \gls{utility} is able to genereate \glspl{dissector} for all \gls{header}-files in a folder, with configuration\\
	\midrule
	\multirow{2}{*}{Execution} & 1. Feed the \gls{utility} the name of the two folders with \gls{header}-files and configuration-files. \\
	& 2. Read output from the \gls{utility} \\
	\midrule
	Expected result  & 2. The \gls{utility} should provide the user with the amount of \gls{header} files processed and the number of \glspl{dissector} created. It should also provide the user with error messages for the \gls{header} and configuration files it was unable to run  \\
	\bottomrule
\end{tabularx}}
\end{table}

\begin{table}[ht] \footnotesize \center
\caption{Test case TID16}\label{tab:TID16}
\noindent\makebox[\textwidth]{%
\begin{tabularx}{1.2\textwidth}{l X}
	\toprule
	Header & Description \\
	\midrule
	Description & Supporting custom \Gls{lua} configuration  \\
	Tester & Sondre Mannsverk \\
	Prerequisites & The \gls{utility} has to be up and running \\
	Feature & Test that the \gls{utility} is able to support custom \Gls{lua} files by configuration \\
	\midrule
	\multirow{6}{*}{Execution} & 1. Look at the custom \Gls{lua} file(s) and understand what they do  \\
	& 2. Feed the \gls{utility} with a \Gls{c} \gls{header}-file, that needs to have specific parts of it dissected by		
	custom \Gls{lua} file(s), and a configuration file that specifies what \Gls{lua} file(s) should be used. \\
	& 3. Read the output \\
	& 4. Move the resulting \gls{dissector} into the plugins folder in the personal configuration
	folder of \Gls{wireshark}\\
	& 5. Open the \gls{pcap-file} for this test with \Gls{wireshark} \\
	& 6. Look at the different \gls{struct} \glspl{member} that are displayed in \Gls{wireshark} \\
	\midrule
	\multirow{2}{*}{Expected result}  & 3. The program should provide the user with some text expressing the			success of the \Gls{lua} file generation \\
	& 6. Assert that the custom \Gls{lua} file(s) have affected the display of the \gls{struct} \glspl{member} in the 		
	expected way. \\
	\bottomrule
\end{tabularx}}
\end{table}

\begin{table}[ht] \footnotesize \center
\caption{Test case TID17}\label{tab:TID17}
\noindent\makebox[\textwidth]{%
\begin{tabularx}{1.2\textwidth}{l X}
	\toprule
	Header & Description \\
	\midrule
	Description & Supporting \glspl{union} \\
	Tester & Lars Solvoll Tønder \\
	Prerequisites & The \gls{utility} has to be up and running \\
	Feature &  Test that the \gls{utility} is able to support \Gls{c}-\gls{header} files with \glspl{union} \\
	\midrule
	\multirow{2}{*}{Execution} & 1. Feed the \gls{utility} with the name of a \gls{header} file with \glspl{union} and its configuration file  \\
	& 2. Read output from the \gls{utility} \\
	& 3. Move the resulting \gls{dissector} into the plugins folder in the personal configuration folder of \Gls{wireshark}\\
	& 4. Open the \gls{pcap-file} for this test with \Gls{wireshark}\\
	& 5. Look at the different \gls{struct} \glspl{member} that are displayed in \Gls{wireshark}\\
	\midrule
	Expected result  & 2. The \gls{utility} should provide the user some text expressing the success of the \Gls{lua}-file generation \\
	& 5. The \glspl{union} should be displayed as subtrees with the names of all the \gls{union} \glspl{member} and their values, even the ones that haven't been explicitly set\\
	\bottomrule
\end{tabularx}}
\end{table}

\clearpage

\begin{table}[ht] \footnotesize \center
\caption{Test case TID18}\label{tab:TID18}
\noindent\makebox[\textwidth]{%
\begin{tabularx}{1.2\textwidth}{l X} 
	\toprule
	Header & Description \\
	\midrule
	Description & Support filter and search in \Gls{wireshark}  \\
	Tester & Erik Bergersen \\
	Prerequisites & \Gls{wireshark} has to be up and running with the trailer\_test.lua \\
	Feature & The \gls{dissector} must support \Gls{wireshark}'s built-in filter and search on attributes \\
	\midrule
	\multirow{3}{*}{Execution} & 1. Open the pcap-file, which contains trailer\_test \glspl{packet}.   \\
	& 2. Type the following in the filter-textbox: ''luatructs.message == 66'' and click apply\\
	& 3. Look on the \gls{packet} view. \\
	\midrule
	Expected result  & 3. Only trailer\_test \glspl{packet} will be visible in the \gls{packet} view.\\
	\bottomrule
\end{tabularx}}
\end{table}

\begin{table}[ht] \footnotesize \center
\caption{Test case TID19}\label{tab:TID19}
\noindent\makebox[\textwidth]{%
\begin{tabularx}{1.2\textwidth}{l X}
 	\toprule
  	Header & Description \\
  	\midrule
  	Description &  Support WIN32, \_WIN64, \_sparc etc \\
  	Tester & Lars Solvoll Tønder \\
  	Prerequisites & The \gls{utility} has to be up and running \\
  	Feature &  Test that the \gls{utility} is able to support \Gls{c}-\gls{header} files with platform definition preprocessor macros (e.g. \_WIN32, \_WIN64, \_sparc, etc.)\\
  	\midrule
  	\multirow{5}{*}{Execution} & 1. Feed the \gls{utility} with \Gls{c}-\gls{header} files with platform definition preprocessor macros (e.g. \_WIN32, \_WIN64, \_sparc, etc.). \\
 	& 2. Read output from the \gls{utility} \\
 	& 3. Move the resulting \gls{dissector} into the plugins folder in the personal configuration folder of \Gls{wireshark}\\
 	& 4. Open the \gls{pcap-file} for this test with \Gls{wireshark}\\
 	& 5. Look at the different \gls{struct} \glspl{member} that are displayed in \Gls{wireshark}\\
	\midrule
	\multirow{2}{*}{Expected result}  & 2. The program should provide the user with some text expressing the success of the \Gls{lua}-file generation. \\
	& 5. The different \glspl{struct} should be displayed as having the right \gls{struct} \glspl{member} according to the \Gls{c}-\gls{header} definitions. For example, the \gls{member} defined surronded by \texttt{\gls{ifdef} \_WIN32 ... \#endif} directives should be displayed correctly in WIN32 \gls{struct}. \\
	\bottomrule
\end{tabularx}}
\end{table}

\begin{table}[ht] \footnotesize \center
\caption{Test case TID20}\label{tab:TID20}
\noindent\makebox[\textwidth]{%
\begin{tabularx}{1.2\textwidth}{l X}
	\toprule
	Header & Description \\
	\midrule
	Description & Supporting the use of flags specifying platforms to display \gls{member} values correctly \\
	Tester & Lars Solvoll Tønder \\
	Prerequisites & A \gls{dissector} for dissecting a \gls{header} according to it's configuration, as well as a \gls{pcap-file} for that \gls{header} needs to be in place\\
	Feature &  Test that the \gls{utility} is able to create \glspl{dissector} that support the use of different flags. These flags should specify the originating platform of the \gls{packet} and be used to display the \gls{member} values properly.\\
	\midrule
	\multirow{2}{*}{Execution} & 1. Open a \gls{pcap-file} which includes \glspl{packet} with different flags but the same \glspl{struct} and \gls{member} values just with different sizes according to the originating platform specified in the flags\\
	& 2. Look at how the different \glspl{packet} are displayed in \Gls{wireshark}\\
	\midrule
	Expected result  & 2. The different packages should have the exact same \gls{member} values just with different flag-values \\
	\bottomrule
\end{tabularx}}
\end{table}

\clearpage

\begin{table}[ht] \footnotesize \center
\caption{Test case TID21}\label{tab:TID21}
\noindent\makebox[\textwidth]{%
\begin{tabularx}{1.2\textwidth}{l X} 
	\toprule
	Header & Description \\
	\midrule
	Description & Supporting platforms with different \gls{endian}  \\
	Tester & Erik Bergersen \\
	Prerequisites & There have to exist a \gls{header}-file, config-file and pcap-file to do the test. \\
	Feature & Generate \glspl{dissector} which support both little and big \gls{endian} platforms \\
	\midrule
	\multirow{5}{*}{Execution} & 1. Feed the \gls{utility} with the name of a \Gls{c}-\gls{header} file with a \gls{struct} \\
	& 2. Read the output \\
	& 3. Move the resulting \gls{dissector} into the plugins folder in the personal configuration folder of \Gls{wireshark}\\
	& 4. Open the \gls{pcap-file} for this test with \Gls{wireshark}\\
	& 5. Look at the different \glspl{member} in \Gls{wireshark} that have data types that can be affected by \gls{endianness}. \\
	\midrule
	\multirow{2}{*}{Expected result}  & 2. The program should provide the user with some text expressing the success of the \Gls{lua}-file generation \\
	& 5. The \glspl{packet} sent from platforms with different \gls{endian}, should display the same values.\\
	\bottomrule
\end{tabularx}}
\end{table}

\begin{table}[ht] \footnotesize \center
\caption{Test case TID22}\label{tab:TID22}
\noindent\makebox[\textwidth]{%
\begin{tabularx}{1.2\textwidth}{l X} 
	\toprule
	Header & Description \\
	\midrule
	Description & Supporting alignments  \\
	Tester & Erik Bergersen \\
	Prerequisites & There have to exist a \gls{header}-file, config-file and pcap-file to do the test. \\
	Feature & Genereate \glspl{dissector} on platforms with different alignment \\
	\midrule
	\multirow{5}{*}{Execution} & 1. Feed the \gls{utility} with the name of a \Gls{c}-\gls{header} file with a \gls{struct} \\
	& 2. Read the output \\
	& 3. Move the resulting \gls{dissector} into the plugins folder in the personal configuration folder of \Gls{wireshark}\\
	& 4. Open the \gls{pcap-file} for this test with \Gls{wireshark}\\
	& 5. Look at the different \glspl{packet} from platforms that uses different alignments. \\
	\midrule
	\multirow{2}{*}{Expected result}  & 2. The program should provide the user with some text expressing the success of the \Gls{lua}-file generation \\
	& 5. The \glspl{packet} sent from platforms that uses different alignments, should display equal values.\\
	\bottomrule
\end{tabularx}}
\end{table}

\begin{table}[ht] \footnotesize \center
\caption{Test case TID23}\label{tab:TID23}
\noindent\makebox[\textwidth]{%
\begin{tabularx}{1.2\textwidth}{l X} 
	\toprule
	Header & Description \\
	\midrule
	Description & Handling \Gls{lua} keywords  \\
	Tester & Erik Bergersen \\
	Prerequisites & There have to exist a \gls{header}-file, config-file and pcap-file to do the test. \\
	Feature & The \gls{utility} should support \gls{struct} \glspl{member}, that have equal names as keywords in \Gls{lua} \\
	\midrule
	\multirow{5}{*}{Execution} & 1. Feed the \gls{utility} with the name of a \Gls{c}-\gls{header} file with a \gls{struct} \\
	& 2. Read the output \\
	& 3. Move the resulting \gls{dissector} into the plugins folder in the personal configuration folder of \Gls{wireshark}\\
	& 4. Open the \gls{pcap-file} for this test with \Gls{wireshark}\\
	& 5. Check that the \gls{dissector} work as it should. \\
	\midrule
	\multirow{2}{*}{Expected result}  & 2. The program should provide the user with some text expressing the success of the \Gls{lua}-file generation \\
	& 5. \Gls{wireshark} should not display an error message for invalid \Gls{lua} code.\\
	\bottomrule
\end{tabularx}}
\end{table}

\begin{table}[ht] \footnotesize \center
\caption{Test case TID24}\label{tab:TID24}
\noindent\makebox[\textwidth]{%
\begin{tabularx}{1.2\textwidth}{l X}
	\toprule
	Header & Description \\
	\midrule
	Description & Sprint 3 functionality test \\
	Tester & Lars Solvoll Tønder \\
	Prerequisites & The \gls{utility} has to have been installed on the system as well as the attest testing framework.The file black\_box.py must also be present in the test folder of Csjark \\
	Feature & Checking that the \gls{utility} is able to create a valid \gls{dissector} from \gls{header} files with all of the data types that were to be supported for sprint 3\\
	\midrule
	\multirow{3}{*}{Execution} & 1. Navigate to the test folder inside the folder where CSjark is installed through the terminal or command line \\
	& 2.  type "python -m attest" into the terminal or command line and then press enter  \\
	& 3. Read the output\\
	\midrule
	Expected result  & 3. The user should be presented with some text expressing the failure of 0 assertions  \\
	\bottomrule
\end{tabularx}}
\end{table}

\begin{table}[ht] \footnotesize \center
\caption{Test case TID25}\label{tab:TID25}
\noindent\makebox[\textwidth]{%
\begin{tabularx}{1.2\textwidth}{l X}
	\toprule
	Header & Description \\
	\midrule
	Description & Sprint 2 functionality test \\
	Tester & Lars Solvoll Tønder \\
	Prerequisites & The \gls{utility} has to have been installed on the system as well as the attest testing framework. The file requirements.py must also be present in the test folder of CSjark \\
	Feature & Checking that the \gls{utility} is able to support all of the features required by the customer \\
	\midrule
	\multirow{3}{*}{Execution} & 1. Navigate to the folder where CSjark is installed through the terminal orcommand line \\
	& 2.  type "python -m attest" into the terminal or command line and then press enter  \\
	& 3. Read the output\\
	\midrule
	Expected result  & 3. The user should be presented with some text expressing the failure of 0 assertions  \\
	\bottomrule
\end{tabularx}}
\end{table}

%----------------------
\section{Sprint 4}
%----------------------

\begin{table}[ht] \footnotesize \center
\caption{Test case TID26}\label{tab:TID26}
\noindent\makebox[\textwidth]{%
\begin{tabularx}{1.2\textwidth}{l X}
	\toprule
	Header & Description \\
	\midrule
	Description & Including system-\glspl{header} \\
	Tester & Lars Solvoll Tønder \\
	Prerequisites & The \gls{utility} has to be installed on the system and there has to exist a \gls{pcap-file} which is associated with this test \\
	Feature & Checking that the \gls{utility} is able to support \glspl{header} which use system \glspl{header} \\
	\midrule
	\multirow{5}{*}{Execution} & 1. Feed the \gls{utility} with the name of a \Gls{c}-\gls{header} file that includes system-\glspl{header} and its configuration file \\
	& 2.  Read the output\\
	& 3. Copy the resulting \glspl{dissector} into the plugins folder of the personal configuration in \Gls{wireshark}\\
	& 4. Run \Gls{wireshark} with the \gls{pcap-file} associated with this test\\
	& 5. Look at the resulting \glspl{struct} and \glspl{member} are displayed in \Gls{wireshark}\\
	\midrule
	\multirow{2}{*}{Expected result}
	& 2. The user should be presented with some text expressing the success of generating \glspl{dissector}\\
	& 5. The \glspl{struct} and \gls{struct} \glspl{member} defined in the system \glspl{header} should be displayed as having a value and not just hex data\\
	\bottomrule
\end{tabularx}}
\end{table}

\begin{table}[ht] \footnotesize \center
\caption{Test case TID27}\label{tab:TID27}
\noindent\makebox[\textwidth]{%
\begin{tabularx}{1.2\textwidth}{l X}
	\toprule
	Header & Description \\
	\midrule
	Description & Ignoring \#pragma directives \\
	Tester & Lars Solvoll Tønder \\
	Prerequisites & The \gls{utility} has to be installed on the system and there needs to exist a \gls{pcap-file} which is associated with this test \\
	Feature & Making sure that the \gls{utility} is able to parse \gls{header} files with the \#pragma directive by just ignoring that directive\\
	\midrule
	\multirow{2}{*}{Execution} & 1. Feed the \gls{utility} with the name of a \Gls{c}-\gls{header} file that contains a \#pragma directive and it's configuration file\\
	& 2. Read the output\\
	\midrule
	\multirow{1}{*}{Expected result}
	& 2. The user should be presented with some text expressing the success of generating \glspl{dissector}\\
	\bottomrule
\end{tabularx}}
\end{table}

\begin{table}[ht] \footnotesize \center
\caption{Test case TID28}\label{tab:TID28}
\noindent\makebox[\textwidth]{%
\begin{tabularx}{1.2\textwidth}{l X}
	\toprule
	Header & Description \\
	\midrule
	Description & Improve generated \Gls{lua} output by removing platform prefix\\
	Tester & Lars Solvoll Tønder \\
	Prerequisites & The \gls{utility} has to be installed on the system \\
	Feature & Making sure the \gls{utility} only generates one \gls{dissector} for the \gls{struct} instead of several, but still keeping all of the functionality\\
	\midrule
	\multirow{7}{*}{Execution} & 1. Feed the \gls{utility} with any \Gls{c}-\gls{header} file and it's configration\\
	& 2.  Read the output\\
	& 3. Copy the resulting \glspl{dissector} into the plugins folder of the personal configuration in \Gls{wireshark}\\
	& 4. Run \Gls{wireshark}\\
	& 5. Open the \gls{dissector} tables menu entry from the Internals menu\\
	& 6. Click the luastructs tree entry\\
	& 7. Inspect its contents\\ 
	\midrule
	\multirow{2}{*}{Expected result}
	& 2. The user should be presented with some text expressing the success of generating \glspl{dissector}\\
	& 7. There should only be one tree entry for each \gls{dissector}, not one for each platform as well\\
\end{tabularx}}
\end{table}

\begin{table}[ht] \footnotesize \center
\caption{Test case TID29}\label{tab:TID29}
\noindent\makebox[\textwidth]{%
\begin{tabularx}{1.2\textwidth}{l X}
	\toprule
	Header & Description \\
	\midrule
	Description & Recursive searching of subfolders\\
	Tester & Lars Solvoll Tønder \\
	Prerequisites & The \gls{utility} has to be installed on the system and there needs to exist a folder with folders that all have \gls{header} files in them \\
	Feature & Checking that it is possible for the \gls{utility} to be fed a folder with \gls{header} files that has subfolders which are in turn inspected\\
	\midrule
	\multirow{3}{*}{Execution} & 1. Feed the \gls{utility} with the name of a folder of \gls{header} files which again has subfolders with other \gls{header} files\\
	& 2.  Read the output\\
	& 3. Inspect the generated \Gls{lua} files \\
	\midrule
	\multirow{2}{*}{Expected result}
	& 2. The user should be presented with some text expressing the success of generating \glspl{dissector} for every \gls{header} file in the subfolders\\
	& 3. There should be 1 file for each \gls{struct} contained in the different \gls{header} files located in the \gls{header} folder and its subfolders \\
\end{tabularx}}
\end{table}

\begin{table}[ht] \footnotesize \center
\caption{Test case TID30}\label{tab:TID30}
\noindent\makebox[\textwidth]{%
\begin{tabularx}{1.2\textwidth}{l X}
	\toprule
	Header & Description \\
	\midrule
	Description & Finding include dependencies which are not explicitly set\\
	Tester & Lars Solvoll Tønder \\
	Prerequisites & The \gls{utility} has to be installed on the system \\
	Feature & Check that the \gls{utility} is able to identify include dependencies which are not explicitly set and use that information to parse files over again correctly\\
	\midrule
	\multirow{5}{*}{Execution} & 1. Feed the \gls{utility} with a \Gls{c}-\gls{header} file that has include dependencies which are not explicitly set, and it's configration file\\
	& 2.  Read the output\\
	& 3. Copy the resulting \glspl{dissector} into the plugins folder of the personal configuration in \Gls{wireshark}\\
	& 4. Run \Gls{wireshark} with the \gls{pcap-file} associated with this test\\
	& 5. Inspect the different \glspl{packet}, their \glspl{struct} and \gls{member} values\\
	\midrule
	\multirow{2}{*}{Expected result}
	& 2. The user should be presented with some text expressing the success of generating \glspl{dissector}\\
	& 5. All \glspl{struct} and \gls{member} values should be displayed as having proper values and not just hex data\\
\end{tabularx}}
\end{table}

\begin{table}[ht] \footnotesize \center
\caption{Test case TID31}\label{tab:TID31}
\noindent\makebox[\textwidth]{%
\begin{tabularx}{1.2\textwidth}{l X}
	\toprule
	Header & Description \\
	\midrule
	Description & Pointer support\\
	Tester & Lars Solvoll Tønder \\
	Prerequisites & The \gls{utility} has to be installed on the system \\
	Feature & Checking that the \gls{utility} is able to support the use of pointers in \gls{header} files\\
	\midrule
	\multirow{5}{*}{Execution} & 1. Feed the \gls{utility} with a \Gls{c}-\gls{header} file that has pointers, and it's configration file\\
	& 2.  Read the output\\
	& 3. Copy the resulting \glspl{dissector} into the plugins folder of the personal configuration in \Gls{wireshark}\\
	& 4. Run \Gls{wireshark} with the \gls{pcap-file} associated with this test\\
	& 5. Inspect the different \glspl{packet}, their \glspl{struct} and \gls{member} values\\
	\midrule
	\multirow{2}{*}{Expected result}
	& 2. The user should be presented with some text expressing the success of generating \glspl{dissector}\\
	& 5. All \glspl{struct} and \gls{member} values should be displayed as having proper values and not just hex data\\
\end{tabularx}}
\end{table}

\begin{table}[ht] \footnotesize \center
\caption{Test case TID32}\label{tab:TID32}
\noindent\makebox[\textwidth]{%
\begin{tabularx}{1.2\textwidth}{l X}
	\toprule
	Header & Description \\
	\midrule
	Description & Enums in \glspl{array}\\
	Tester & Lars Solvoll Tønder \\
	Prerequisites & The \gls{utility} has to be installed on the system \\
	Feature & Checking that the \gls{utility} is able to support the use of \glspl{enum} inside of \glspl{array}\\
	\midrule
	\multirow{5}{*}{Execution} & 1. Feed the \gls{utility} with a \Gls{c}-\gls{header} file that has an \gls{array} of \glspl{enum}, and it's configration file\\
	& 2.  Read the output\\
	& 3. Copy the resulting \glspl{dissector} into the plugins folder of the personal configuration in \Gls{wireshark}\\
	& 4. Run \Gls{wireshark} with the \gls{pcap-file} associated with this test\\
	& 5. Inspect the different \glspl{packet}, their \glspl{struct} and \gls{member} values\\
	\midrule
	\multirow{2}{*}{Expected result}
	& 2. The user should be presented with some text expressing the success of generating \glspl{dissector}\\
	& 5. All \glspl{struct} that contain an \gls{array} of \glspl{enum} should be displayed as containing \glspl{enum} instead of just integers\\
\end{tabularx}}
\end{table}

\begin{table}[ht] \footnotesize \center
\caption{Test case TID33}\label{tab:TID33}
\noindent\makebox[\textwidth]{%
\begin{tabularx}{1.2\textwidth}{l X}
	\toprule
	Header & Description \\
	\midrule
	Description & Supporting \gls{define} as a command line argument\\
	Tester & Lars Solvoll Tønder \\
	Prerequisites & The \gls{utility} has to be installed on the system \\
	Feature & Checking that it is possible to feed the \gls{utility} with a \gls{define} argument and have it force pre-processor to add a corresponding argument to the \gls{header} files it is processing\\
	\midrule
	\multirow{5}{*}{Execution} & 1. Feed the \gls{utility} with a \Gls{c}-\gls{header} file, it's configration file and a \gls{define} directive\\
	& 2.  Read the output\\
	& 3. Copy the resulting \glspl{dissector} into the plugins folder of the personal configuration in \Gls{wireshark}\\
	& 4. Run \Gls{wireshark} with the \gls{pcap-file} associated with this test\\
	& 5. Inspect the different \glspl{packet}, their \glspl{struct} and \gls{member} values\\
	\midrule
	\multirow{2}{*}{Expected result}
	& 2. The user should be presented with some text expressing the success of generating \glspl{dissector}\\
	& 5. All \glspl{struct} and their \gls{member} values should be displayed as having proper values and not just hex data\\
\end{tabularx}}
\end{table}

\begin{table}[ht] \footnotesize \center
\caption{Test case TID34}\label{tab:TID34}
\noindent\makebox[\textwidth]{%
\begin{tabularx}{1.2\textwidth}{l X}
	\toprule
	Header & Description \\
	\midrule
	Description & Multiple message ID's for one \gls{dissector}\\
	Tester & Lars Solvoll Tønder \\
	Prerequisites & The \gls{utility} has to be installed on the system \\
	Feature & Checking that the \gls{utility} supports having more than one message ID per \gls{dissector}\\
	\midrule
	\multirow{2}{*}{Execution} & 1. Feed the \gls{utility} with a \Gls{c}-\gls{header} file and it's configuration file which includes more than one message ID\\
	& 2.  Read the output\\ 
	\midrule
	\multirow{1}{*}{Expected result}
	& 2. The user should be presented with some text expressing the success of generating \glspl{dissector}\\
\end{tabularx}}
\end{table}

\begin{table}[ht] \footnotesize \center
\caption{Test case TID35}\label{tab:TID35}
\noindent\makebox[\textwidth]{%
\begin{tabularx}{1.2\textwidth}{l X}
	\toprule
	Header & Description \\
	\midrule
	Description & Allowing configuration for unknown \glspl{struct}\\
	Tester & Lars Solvoll Tønder \\
	Prerequisites & The \gls{utility} has to be installed on the system \\
	Feature & Checking that the \gls{utility} supports being able to configure the size of unknown \glspl{struct}\\
	\midrule
	\multirow{5}{*}{Execution} & 1. Feed the \gls{utility} with a \Gls{c}-\gls{header} file that has unparseable \glspl{member} and it's configuration file which includes the size of the \gls{struct} itself\\
	& 2.  Read the output\\
	& 3. Copy the resulting \glspl{dissector} into the plugins folder of the personal configuration in \Gls{wireshark}\\
	& 4. Open \Gls{wireshark} with the \gls{pcap-file} associated with this test\\
	& 5. Inspect the different \glspl{packet}, their \glspl{struct} and \gls{member} values\\ 
	\midrule
	\multirow{1}{*}{Expected result}
	& 2. The user should be presented with some text expressing the success of generating \glspl{dissector}\\
	& 5. All of the \glspl{member} of each \glspl{packet} should have proper values except for the unparseable \glspl{member} which should only be displayed as containing hex data\\
\end{tabularx}}
\end{table}

\clearpage

\begin{table}[ht] \footnotesize \center
\caption{Test case TID36}\label{tab:TID36}
\noindent\makebox[\textwidth]{%
\begin{tabularx}{1.2\textwidth}{l X}
	\toprule
	Header & Description \\
	\midrule
	Description & Autogenerating configuration files for \glspl{struct} that has no config file of their own\\
	Tester & Lars Solvoll Tønder \\
	Prerequisites & The \gls{utility} has to be installed on the system \\
	Feature & Checking that the \gls{utility} is able to create template configuration files for all \glspl{struct} that does not currently have a configuration\\
	\midrule
	\multirow{3}{*}{Execution} & 1. Feed the \gls{utility} with a \Gls{c}-\gls{header} file which has several \glspl{struct} without any configuration files \\
	& 2.  Open the configuration folder where CSjark is installed\\
	& 3. Inspect the configuration files \\
	\midrule
	\multirow{1}{*}{Expected result}
	& 3. There should now be one configuration file present for each \gls{struct} in the \Gls{c}-\gls{header} file that has an empty template for filling in values for configuration\\
\end{tabularx}}
\end{table}

\begin{table}[ht] \footnotesize \center
\caption{Test case TID37}\label{tab:TID37}
\noindent\makebox[\textwidth]{%
\begin{tabularx}{1.2\textwidth}{l X}
	\toprule
	Header & Description \\
	\midrule
	Description & Only generating \glspl{dissector} for \glspl{struct} with a valid ID\\
	Tester & Lars Solvoll Tønder \\
	Prerequisites & The \gls{utility} has to be installed on the system \\
	Feature & Making sure that the \gls{utility} only creates \glspl{dissector} for files that have a valid ID specified in its configuration\\
	\midrule
	\multirow{2}{*}{Execution} & 1. Feed the \gls{utility} with a \Gls{c}-\gls{header} file and it's configuration file that has no valid ID\\
	& 2.  Read the output\\
	\midrule
	\multirow{1}{*}{Expected result}
	& 2. The user should be presented with a message saying that no \glspl{dissector} were created and why\\
\end{tabularx}}
\end{table}

\begin{table}[ht] \footnotesize \center
\caption{Test case TID38}\label{tab:TID38}
\noindent\makebox[\textwidth]{%
\begin{tabularx}{1.2\textwidth}{l X}
	\toprule
	Header & Description \\
	\midrule
	Description & Guessing \glspl{dissector} from \gls{packet} size\\
	Tester & Lars Solvoll Tønder \\
	Prerequisites & The \gls{utility} has to be installed on the system and \Gls{wireshark} has to have been loaded with a \gls{dissector} for a \gls{struct} of the same size as the one associated with this test \\
	Feature & Making sure the \gls{utility} is able to guess which \gls{dissector} to use based on \gls{packet} size if there are no \glspl{dissector} specified for the \gls{packet}\\
	\midrule
	\multirow{2}{*}{Execution} & 1.Start \Gls{wireshark} with the \gls{pcap-file} associated with this test\\
	& 2.  Inspect the different \glspl{packet}, their \glspl{struct} and \gls{member} values\\
	\midrule
	\multirow{1}{*}{Expected result}
	& 2. All of the \glspl{packet} should contain a \gls{struct} with different \glspl{member} and values instead of just raw hex data\\
\end{tabularx}}
\end{table}

\begin{table}[ht] \footnotesize \center
\caption{Test case TID39}\label{tab:TID39}
\noindent\makebox[\textwidth]{%
\begin{tabularx}{1.2\textwidth}{l X}
	\toprule
	Header & Description \\
	\midrule
	Description & Invalid \gls{header}\\
	Tester & Lars Solvoll Tønder \\
	Prerequisites & The \gls{utility} has to be installed on the system\\
	Feature & Making sure the \gls{utility} crashes if it receives an invalid \gls{header}\\
	\midrule
	\multirow{2}{*}{Execution} & 1. Feed the \gls{utility} with an invalid \gls{header} file\\
	& 2.  Read the output\\
	\midrule
	\multirow{1}{*}{Expected result}
	& 2. The \gls{utility} should crash and give an error message explaining why it crashed\\
\end{tabularx}}
\end{table}

\begin{table}[ht] \footnotesize \center
\caption{Test case TID40}\label{tab:TID40}
\noindent\makebox[\textwidth]{%
\begin{tabularx}{1.2\textwidth}{l X}
	\toprule
	Header & Description \\
	\midrule
	Description & Invalid \gls{header} during \gls{batch mode}\\
	Tester & Lars Solvoll Tønder \\
	Prerequisites & The \gls{utility} has to be installed on the system\\
	Feature & Making sure the \gls{utility} skips \glspl{header} it is unable to parse during \gls{batch processing}\\
	\midrule
	\multirow{2}{*}{Execution} & 1. Feed the \gls{utility} with a folder containing several invalid \gls{header} files\\
	& 2.  Read the output\\
	\midrule
	\multirow{1}{*}{Expected result}
	& 2. The \gls{utility} should skip the files it is unable to parse and present the user with a message saying why it skipped the files it was unable to parse\\
\end{tabularx}}
\end{table}

\begin{table}[ht] \footnotesize \center
\caption{Test case TID41}\label{tab:TID41}
\noindent\makebox[\textwidth]{%
\begin{tabularx}{1.2\textwidth}{l X}
	\toprule
	Header & Description \\
	\midrule
	Description & Ambiguous \gls{struct} IDs \\
	Tester & Lars Solvoll Tønder \\
	Prerequisites & The \gls{utility} has to be installed on the system\\
	Feature & Making sure the \gls{utility} crashes if there are several \glspl{struct} that all have been configured to using the same ID\\
	\midrule
	\multirow{2}{*}{Execution} & 1. Feed the \gls{utility} with a pair of \glspl{header} and configuration files that both have the same \gls{struct} ID\\
	& 2.  Read the output\\
	\midrule
	\multirow{1}{*}{Expected result}
	& 2. The \gls{utility} should present the user with an error message saying that there exists 2 \glspl{struct} which have been configured to using the same ID\\
\end{tabularx}}
\end{table}

\begin{table}[ht] \footnotesize \center
\caption{Test case TID42}\label{tab:TID42}
\noindent\makebox[\textwidth]{%
\begin{tabularx}{1.2\textwidth}{l X}
	\toprule
	Header & Description \\
	\midrule
	Description & Ambiguous platform IDs \\
	Tester & Lars Solvoll Tønder \\
	Prerequisites & The \gls{utility} has to be installed on the system\\
	Feature & Making sure the \gls{utility} crashes if there are several platforms that all share the same ID\\
	\midrule
	\multirow{3}{*}{Execution} & 1.Add another platform platform.py which has the same platform ID as one of the previous ones\\
	& 2.  Feed the \gls{utility} with any \gls{header} file and its config\\
	& 3. Read the output\\
	\midrule
	\multirow{1}{*}{Expected result}
	& 3. The \gls{utility} should present the user with an error message saying that there exists two platforms with the same name\\
\end{tabularx}}
\end{table}

\begin{table}[ht] \footnotesize \center
\caption{Test case TID43}\label{tab:TID43}
\noindent\makebox[\textwidth]{%
\begin{tabularx}{1.2\textwidth}{l X}
	\toprule
	Header & Description \\
	\midrule
	Description & Running the \gls{utility} on \Gls{Windows}\\
	Tester & Lars Solvoll Tønder \\
	Prerequisites & The \gls{utility} has to be installed on a \Gls{Windows} system\\
	Feature & Making sure the \gls{utility} runs on the latest version of \Gls{Windows} as per 24.11.2011\\
	\midrule
	\multirow{2}{*}{Execution} & 1. Feed the \gls{utility} with any \Gls{c}-\gls{header} file and its config\\
	& 2. Read the output\\
	\midrule
	\multirow{1}{*}{Expected result}
	& 2. The \gls{utility} should present the user with some text expressing the success of creating a \gls{dissector} for each of the \glspl{struct} inside the \gls{header} file\\
\end{tabularx}}
\end{table}

\begin{table}[ht] \footnotesize \center
\caption{Test case TID44}\label{tab:TID44}
\noindent\makebox[\textwidth]{%
\begin{tabularx}{1.2\textwidth}{l X}
	\toprule
	Header & Description \\
	\midrule
	Description & Running the \gls{utility} on \Gls{Solaris}\\
	Tester & Lars Solvoll Tønder \\
	Prerequisites & The \gls{utility} has to be installed on a \Gls{Solaris} system\\
	Feature & Making sure the \gls{utility} runs on the latest version of \Gls{Solaris} as per 24.11.2011\\
	\midrule
	\multirow{2}{*}{Execution} & 1. Feed the \gls{utility} with any \Gls{c}-\gls{header} file and its config\\
	& 2. Read the output\\
	\midrule
	\multirow{1}{*}{Expected result}
	& 2. The \gls{utility} should present the user with some text expressing the success of creating a \gls{dissector} for each of the \glspl{struct} inside the \gls{header} file\\
\end{tabularx}}
\end{table}

\begin{table}[ht] \footnotesize \center
\caption{Test case TID45}\label{tab:TID45}
\noindent\makebox[\textwidth]{%
\begin{tabularx}{1.2\textwidth}{l X}
	\toprule
	Header & Description \\
	\midrule
	Description & Running the \glspl{dissector} on \Gls{Solaris}\\
	Tester & Lars Solvoll Tønder \\
	Prerequisites & \Gls{wireshark} has to have been installed with the \glspl{dissector} associated with this test\\
	Feature & Making sure the \glspl{dissector} runs on the latest version of \Gls{Solaris} as per 24.11.2011\\
	\midrule
	\multirow{2}{*}{Execution} & 1. Run \Gls{wireshark} with the \gls{pcap-file} associated with this test\\
	& 2. Inspect the different \glspl{packet}, their \glspl{struct} and \gls{member} values\\
	\midrule
	\multirow{1}{*}{Expected result}
	& 2. The \glspl{packet} should be displayed as containing \glspl{struct} with \glspl{member} that all have proper values and not just hex data\\
\end{tabularx}}
\end{table}

\begin{table}[ht] \footnotesize \center
\caption{Test case TID46}\label{tab:TID46}
\noindent\makebox[\textwidth]{%
\begin{tabularx}{1.2\textwidth}{l X}
	\toprule
	Header & Description \\
	\midrule
	Description & Running the \glspl{dissector} on \Gls{Windows}\\
	Tester & Lars Solvoll Tønder \\
	Prerequisites & \Gls{wireshark} has to have been installed with the \glspl{dissector} associated with this test\\
	Feature & Making sure the \glspl{dissector} runs on the latest version of \Gls{Solaris} as per 24.11.2011\\
	\midrule
	\multirow{2}{*}{Execution} & 1. Run \Gls{wireshark} with the \gls{pcap-file} associated with this test\\
	& 2. Inspect the different \glspl{packet}, their \glspl{struct} and \gls{member} values\\
	\midrule
	\multirow{1}{*}{Expected result}
	& 2. The \glspl{packet} should be displayed as containing \glspl{struct} with \glspl{member} that all have proper values and not just hex data\\
\end{tabularx}}
\end{table}

\begin{table}[ht] \footnotesize \center
\caption{Test case TID47}\label{tab:TID47}
\noindent\makebox[\textwidth]{%
\begin{tabularx}{1.2\textwidth}{l X}
	\toprule
	Header & Description \\
	\midrule
	Description & Quality of user documentation\\
	Tester & Lars Solvoll Tønder \\
	Prerequisites & The user documentation has to have been written\\
	\midrule
	\multirow{2}{*}{Execution} & 1. Read the user documentation for one hour\\
	& 2. Try using the utility to generate a dissector for a header file\\
	\midrule
	\multirow{1}{*}{Expected result}
	& 2. The user should be able to generate a dissector configured to display the values from a given c-struct properly\\
\end{tabularx}}
\end{table}