%=================
\chapter{Sprint 4}
%=================


%------------------------
\section{Sprint Planning}
%------------------------
The fourth sprint will be the last iteration in this project. Testing and bugfixing will take most of the sprint work hours, because it is very important to make the utility work properly on Thales source code. Now the utility fail on most of the structs: Stig tried to generate dissectors for 580 structs, where only 4 of them succeeded. It will probably just be some small tweaks needed, to make it work for most of the structs.

We know that some of our earlier implementation do not work as intended or is incomplete. The customer has given us feedback on the fulfilled requirements, and pointed out what needs to be improved. These are important work items of this sprint backlog MAKE REF. In addition we got new requirements that would be nice if we implemented, some which will be convenient for the customer to have. 

There will not be another sprint after this one, so we can not postpone any work items. Because of this, we had to make some of the new requirements from the customer optional, and also telling the customer that we will probably not be able complete all their desires. The last sprint we had too many tasks and too little time, but we still managed to cover almost everything in the end. This sprint we plan to use close to 350 hours, but it is still likely that we will use more and are prepared for that. It will certainly pop up new tasks and unforeseen bugs that we have to fix.

\subsection{Duration}
%-----------------------
The sprint started with the planning meeting the 2nd of November and our work started the following day. The sprint duration is 14 days, and will end the 15th of November with a review meeting. 

\subsection{Sprint Goal}
%-----------------------





For the third sprint the team will update CSjark to version 0.3 which will extend the \gls{utility} so that it contains the complete functionality requested by the customer at this phase of the project. In this sprint we will pick all of the current requirements from the product backlog as all of the underlying functionality needed for them are already in place from the previous sprints. This means that we will also aim to create a draft of the final design of the system during the sprint.

The most important function that is going to be implemented in this sprint is being able to display packets from different originating platforms properly. This will be implemented by having every \gls{packet} contain a flag specifying their originating platform, and by having our \glspl{dissector} use this flag value to influence how it handles the data in the \gls{packet}.

\subsection{Back Log}
%--------------------


%----------------------
\section{System Design}
%----------------------


%-----------------------
\section{Implementation}
%-----------------------


%-----------------------
\section{Sprint Testing}
%-----------------------


%--------------------------
\section{Customer Feedback}
%--------------------------


%--------------------------
\section{Sprint Evaluation}
%--------------------------


