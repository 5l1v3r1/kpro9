
%==============
\chapter{Test Cases}
%==============


%-----------------------
\section{Sprint 1 Tests}
%-----------------------
Test cases for sprint 1 is \autoref{tab:TID01}-\ref{tab:TID07}. 

\begin{table}[ht] \footnotesize \center
\caption{Test case TID01}\label{tab:TID01}
\noindent\makebox[\textwidth]{%
\begin{tabularx}{1.2\textwidth}{l X}
	\toprule
	Header & Description \\
	\midrule
	Description & Supporting parameters for c-header file \\
	Tester & Lars Solvoll Tønder \\
	Prerequisites & The program has to have been compiled on the system \\
	Feature & Test that we are able to feed the solution with a c-header file and have it get dissected \\
	\midrule
	\multirow{3}{*}{Execution} & 1. Start the program \\
	& 2. Write the name of the c-header file in the command line \\
	& 3. Read the output given by the program \\
	\midrule
	\multirow{2}{*}{Expected result} & 1. The program should start up without any errors and present the user with a command line interface \\
	& 3. The user should be presented with some text expressing the success of the LUA-file generation \\
	\bottomrule
\end{tabularx}}
\end{table}

\begin{table}[ht] \footnotesize \center
\caption{Test case TID02}\label{tab:TID02}
\noindent\makebox[\textwidth]{%
\begin{tabularx}{1.2\textwidth}{l X}
	\toprule
	Header & Description \\
	\midrule
	Description & Supporting basic data types \\
	Tester & Lars Solvoll Tønder \\
	Prerequisites & The program has to have been started \\
	Feature & Test that our utility will be able to make a dissectors for C-header files including the following basic data types: int, float, char and boolean \\
	\midrule
	\multirow{2}{*}{Execution} & 1. Write the name of a c-header file which includes the aforementioned basic data types \\
	& 2. Read the output given by the program \\ 
	\midrule
	Expected result & 2. The program should provide the user with some text expressing the success of the LUA-file generation \\
	\bottomrule
\end{tabularx}}
\end{table}

\begin{table}[ht] \footnotesize \center
\caption{Test case TID03}\label{tab:TID03}
\noindent\makebox[\textwidth]{%
\begin{tabularx}{1.2\textwidth}{l X}
	\toprule
	Header & Description \\
	\midrule
	Description & Displaying simple structs \\
	Tester & Lars Solvoll Tønder \\
	Prerequisites & The utility has already made a dissector \\
	Feature & Test that our utility is able to generate dissectors that displays simple structs \\
	\midrule
	\multirow{3}{*}{Execution} & 1. Open wireshark and run the dissector \\
	& 2. Run the dissector on some captured data of simple structs \\
	& 3. Read the output \\
	\midrule
	\multirow{2}{*}{Expected result} & 1. Wireshark should be able to load the dissector without any errors \\
	& 3. Wireshark should display the data inside the structs sent in the capture data \\
	\bottomrule
\end{tabularx}}
\end{table}

\begin{table}[ht] \footnotesize \center
\caption{Test case TID04}\label{tab:TID04}
\noindent\makebox[\textwidth]{%
\begin{tabularx}{1.2\textwidth}{l X}
	\toprule
	Header & Description \\
	\midrule
	Description & Supporting \#include \\
	Tester & Lars Solvoll Tønder \\
	Prerequisites & The utility has to be up and running \\
	Feature & Test that our utility supports c-header files with the \#include directive \\
	\midrule
	\multirow{2}{*}{Execution} & 1. Write the name of a C-header file with an \#include directive \\
	& 2. Read the output \\
	\midrule
	Expected result & 2. The program should provide the user with some text expressing the success of the LUA-file generation \\
	\bottomrule
\end{tabularx}}
\end{table}

\begin{table}[ht] \footnotesize \center
\caption{Test case TID05}\label{tab:TID05}
\noindent\makebox[\textwidth]{%
\begin{tabularx}{1.2\textwidth}{l X}
	\toprule
	Header & Description \\
	\midrule
	Description & Supporting \#define and \#if \\
	Tester & Lars Solvoll Tønder \\
	Prerequisites & The utility has to be up and running \\
	Feature & Test that our utility supports c-header files with \#define and \#if directives \\
	\midrule
	\multirow{2}{*}{Execution} & 1. Write the name of a C-header file with a \#define and \#if directives \\
	& 2. Read the output \\
	\midrule
	Expected result & 2. The program should provide the user with some text expressing the success of the LUA-file generation \\
	\bottomrule
\end{tabularx}}
\end{table}

\begin{table}[ht] \footnotesize \center
\caption{Test case TID06}\label{tab:TID06}
\noindent\makebox[\textwidth]{%
\begin{tabularx}{1.2\textwidth}{l X}
	\toprule
	Header & Description \\
	\midrule
	Description & Supporting configuration files \\
	Tester & Lars Solvoll Tønder \\
	Prerequisites & The utility has to be up and running \\
	Feature & Test that our utility supports reading data from a configuration file \\
	\midrule
	\multirow{2}{*}{Execution} & 1. Feed the utility with the name of a config file \\
	& 2. Read the output \\
	\midrule
	Expected result & 2. The program should provide the user with some text expressing either the success of reading the file, or the failure of reading the file due to any errors in the config-file itself\\
	\bottomrule
\end{tabularx}}
\end{table}

\begin{table}[ht] \footnotesize \center
\caption{Test case TID07}\label{tab:TID07}
\noindent\makebox[\textwidth]{%
\begin{tabularx}{1.2\textwidth}{l X}
	\toprule
	Header & Description \\
	\midrule
	Description & Recognizing invalid values \\
	Tester & Lars Solvoll Tønder \\
	Prerequisites & The utility must have already loaded a configuration file and generated a dissector for a given header file\\
	Feature & Test that our utility recognizes invalid values for struct members specified in the config file. \\
	\midrule
	\multirow{3}{*}{Execution} & 1. Feed the utility with the name of a header and a config-file, where the config file sets restrictions on the members of the header file. \\
	& 2. Use the resulting dissector on some capture data for wireshark that contains struct members with invalid values. \\
	& 3. Read the displayed data generated by wireshark \\
	\midrule
	Expected result & 3. Wireshark should display the struct members with invalid values as invalid \\
	\bottomrule
\end{tabularx}}
\end{table}


%-----------------------
\section{Sprint 2 Tests}
%-----------------------
Test cases for sprint 2 is \autoref{tab:TID08}-\ref{tab:TID14}. 

\begin{table}[ht] \footnotesize \center
\caption{Test case TID08}\label{tab:TID08}
\noindent\makebox[\textwidth]{%
\begin{tabularx}{1.2\textwidth}{l X}
	\toprule
	Header & Description \\
	\midrule
	Description & Supporting members of type enum \\
	Tester & Lars Solvoll Tønder \\
	Prerequisites & The utility has to be up and running \\
	Feature & Test that the utility iis able to support c-header files with enums \\
	\midrule
	\multirow{5}{*}{Execution} & 1. Feed the utility with the name of a c-header file which includes a struct using enums and its configuration file \\
	& 2. Read the output \\
	& 3. Move the resulting dissector into the plugins folder in the personal configuration folder of wireshark\\
	& 4. Open the pcap file for this test with wireshark\\
	& 5. Look at the different packets and struct members that are displayed in wireshark\\
	\midrule
	\multirow{2}{*}{Expected result } & 2. The program should provide the user with some text expressing the success of the LUA-file generation \\ 
	\addlinespace
	& 5 The different packets should be displayed as having structs with enums, showing the value of the enum by its name and value\\
	\bottomrule
\end{tabularx}}
\end{table}

\begin{table}[ht] \footnotesize \center
\caption{Test case TID09}\label{tab:TID09}
\noindent\makebox[\textwidth]{%
\begin{tabularx}{1.2\textwidth}{l X}
	\toprule
	Header & Description \\
	\midrule
	Description & Supporting members of type array \\
	Tester & Lars Solvoll Tønder \\
	Prerequisites & The utility has to be up and running \\
	Feature & Test that the utility iis able to support c-header files with arrays \\
	\midrule
	\multirow{5}{*}{Execution} & 1. Feed the utility with the name of a c-header file which includes a struct using arrays \\
	& 2. Read the output \\
	& 3. Move the resulting dissector into the plugins folder in the personal configuration folder of wireshark\\
	& 4. Open the pcap file for this test with wireshark\\
	& 5. Look at the different packets and struct members that are displayed in wireshark\\
	\midrule
	\multirow{5}{*}{Expected result} & 2. The program should provide the user with some text expressing the success of the LUA-file generation \\
	& 5 The different packets should be displayed as having structs with arrays, showing the values of the cells in the array. Multidimensional arrays should also be displayed as being arrays with subtrees of other arrays. This should be indicated by the multidimensional arrays having a clickable "+" box to the left which when pressed shows the values of the cells of the inner arrays \\
	\bottomrule
\end{tabularx}}
\end{table}

\begin{table}[ht] \footnotesize \center
\caption{Test case TID10}\label{tab:TID10}
\noindent\makebox[\textwidth]{%
\begin{tabularx}{1.2\textwidth}{l X}
	\toprule
	Header & Description \\
	\midrule
	Description & Supporting the display of structs within structs \\
	Tester & Erik Bergersen \\
	Prerequisites & The utility has to be up and running \\
	Feature & Test that the utility iis able to support c-header files with structs that have other struct members  and display it properly in wireshark\\
	\midrule
	\multirow{5}{*}{Execution} & 1. Feed the utility with the name of a c-header file which includes a struct with another struct member \\
	& 2. Read the output \\
	& 3. Move the resulting dissector into the plugins folder in the personal configuration folder of wireshark\\
	& 4. Open the pcap file for this test with wireshark\\
	& 5. Look at the different struct members that are displayed in wireshark\\
	\midrule
	\multirow{2}{*}{Expected result} & 2. The program should provide the user with some text expressing the success of the LUA-file generation\\
	& 5. The struct members within structs should be displayed as subtrees, indicated by having a clickable "+" box to the left whch when clicked, displayes the contents of the given struct\\
	\bottomrule
\end{tabularx}}
\end{table}

\begin{table}[ht] \footnotesize \center
\caption{Test case TID11}\label{tab:TID11}
\noindent\makebox[\textwidth]{%
\begin{tabularx}{1.2\textwidth}{l X}
	\toprule
	Header & Description \\
	\midrule
	Description & Supporting enumerated named values \\
	Tester & Erik Bergersen \\
	Prerequisites & The utility has to be up and running \\
	Feature & Test that the utility iis able to support c-header files which uses integers as enums without declaring them as enums \\
	\midrule
	\multirow{5}{*}{Execution} & 1. Feed the utility with the name of a c-header file which includes enumerated named values \\
	& 2. Read the output \\
	& 3. Move the resulting dissector into the plugins folder in the personal configuration folder of wireshark\\
	& 4. Open the pcap file for this test with wireshark\\
	& 5. Look at the different struct members that are displayed in wireshark\\
	\midrule
	\multirow{2}{*}{Expected result}  & 2. The program should provide the user with some text expressing the success of the LUA-file generation \\
	\addlinespace
	& 5. The different packets should be displayed as containing enumerated named values expressed by their name and not value\\
	\bottomrule
\end{tabularx}}
\end{table}

\begin{table}[ht] \footnotesize \center
\caption{Test case TID12}\label{tab:TID12}
\noindent\makebox[\textwidth]{%
\begin{tabularx}{1.2\textwidth}{l X}
	\toprule
	Header & Description \\
	\midrule
	Description & Supporting bit strings \\
	Tester & Erik Bergersen \\
	Prerequisites & The utility has to be up and running \\
	Feature & Test that the utility iis able to support c-header files with bit strings \\
	\midrule
	\multirow{5}{*}{Execution} & 1. Feed the utility with the name of a c-header file with a struct using bit strings \\
	& 2. Read the output \\
	& 3. Move the resulting dissector into the plugins folder in the personal configuration folder of wireshark\\
	& 4. Open the pcap file for this test with wireshark\\
	& 5. Look at the different struct members that are displayed in wireshark\\
	\midrule
	\multirow{2}{*}{Expected result} & 2. The program should provide the user with some text expressing the success of the LUA-file generation \\
	& 5. The packets should be displayed as containing bit strings. Bit strings should be displayed as subtrees indicated by a clickable "+" box to the left which shows the values of the different bits in the bit string when pressed\\
	\bottomrule
\end{tabularx}}
\end{table}

\begin{table}[ht] \footnotesize \center
\caption{Test case TID13}\label{tab:TID13}
\noindent\makebox[\textwidth]{%
\begin{tabularx}{1.2\textwidth}{l X}
	\toprule
	Header & Description \\
	\midrule
	Description & Supporting structs with various trailers \\
	Tester & Erik Bergersen \\
	Prerequisites & The utility has to be up and running \\
	Feature & Test that the utility iis able to support c-header files with trailers \\
	\midrule
	\multirow{5}{*}{Execution} & 1. Feed the utility with the name of a c-header file with a struct that has a trailer \\
	& 2. Read the output \\
	& 3. Move the resulting dissector into the plugins folder in the personal configuration folder of wireshark\\
	& 4. Open the pcap file for this test with wireshark\\
	& 5. Look at the different struct members that are displayed in wireshark\\
	\midrule
	\multirow{2}{*}{Expected result}  & 2. The program should provide the user with some text expressing the success of the LUA-file generation \\
	& 5. The packets should be display the data inside the struct, and the data from the trailer should be displayed below\\
	\bottomrule
\end{tabularx}}
\end{table}

\begin{table}[ht] \footnotesize \center
\caption{Test case TID14}\label{tab:TID14}
\noindent\makebox[\textwidth]{%
\begin{tabularx}{1.2\textwidth}{l X}
	\toprule
	Header & Description \\
	\midrule
	Description & Sprint 2 functionality test \\
	Tester & Lars Solvoll Tønder \\
	Prerequisites & The utility has to have been installed on the system as well as the attest testing framework \\
	Feature & Checking that the utility is able to create a valid dissector from header files with all of the data types that were to be supported for sprint 2, including: trailers, bit strings, enumerated named values, structs within structs and arrays \\
	\midrule
	\multirow{3}{*}{Execution} & 1. Navigate to the folder where CSjark is installed through the terminal orcommand line \\
	& 2.  type "python -m attest" into the terminal or command line and then press enter  \\
	& 3. Read the output\\
	\midrule
	Expected result  & 3. The user should be presented with some text expressing the failure of 0 assertions  \\
	\bottomrule
\end{tabularx}}
\end{table}

%------------------------------
\section{Sprint 3 Tests}
%-----------------------------
Test cases for sprint 3 is \autoref{tab:TID15}-\ref{tab:TID23}. 

\begin{table}[ht] \footnotesize \center
\caption{Test case TID15}\label{tab:TID15}
\noindent\makebox[\textwidth]{%
\begin{tabularx}{1.2\textwidth}{l X}
	\toprule
	Header & Description \\
	\midrule
	Description & Support batch mode of C header and configuration files \\
	Tester & Lars Solvoll Tønder \\
	Prerequisites & The utility has to be up and running \\
	Feature &  Test that the utility is able to genereate dissectors for all header-files in a folder, with configuration\\
	\midrule
	\multirow{2}{*}{Execution} & 1. Feed the utility the name of the two folders with header-files and configuration-files. \\
	& 2. Read output from the utility \\
	\midrule
	Expected result  & 2. The utility should provide the user with the amount of header files processed and the number of dissectors created. It should also provide the user with error messages for the header and configuration files it was unable to run  \\
	\bottomrule
\end{tabularx}}
\end{table}

\begin{table}[ht] \footnotesize \center
\caption{Test case TID16}\label{tab:TID16}
\noindent\makebox[\textwidth]{%
\begin{tabularx}{1.2\textwidth}{l X}
	\toprule
	Header & Description \\
	\midrule
	Description & Supporting custom Lua configuration  \\
	Tester & Sondre Mannsverk \\
	Prerequisites & The utility has to be up and running \\
	Feature & Test that the utility is able to support custom Lua files by configuration \\
	\midrule
	\multirow{6}{*}{Execution} & 1. Look at the custom Lua file(s) and understand what they do  \\
	& 2. Feed the utility with a C header-file, that needs to have specific parts of it dissected by		
	custom Lua file(s), and a configuration file that specifies what Lua file(s) should be used. \\
	& 3. Read the output \\
	& 4. Move the resulting dissector into the plugins folder in the personal configuration
	folder of wireshark\\
	& 5. Open the pcap file for this test with wireshark \\
	& 6. Look at the different struct members that are displayed in wireshark \\
	\midrule
	\multirow{2}{*}{Expected result}  & 3.  The program should provide the user with some text expressing the			success of the Lua file generation \\
	& 6. Assert that the custom Lua file(s) have affected the display of the struct members in the 		
	expected way. \\
	\bottomrule
\end{tabularx}}
\end{table}

\begin{table}[ht] \footnotesize \center
\caption{Test case TID17}\label{tab:TID17}
\noindent\makebox[\textwidth]{%
\begin{tabularx}{1.2\textwidth}{l X}
	\toprule
	Header & Description \\
	\midrule
	Description & Supporting unions \\
	Tester & Lars Solvoll Tønder \\
	Prerequisites & The utility has to be up and running \\
	Feature &  Test that the utility is able to support c-header files with unions \\
	\midrule
	\multirow{2}{*}{Execution} & 1. Feed the utility with the name of a header file with unions and its configuration file  \\
	& 2. Read output from the utility \\
	& 3. Move the resulting dissector into the plugins folder in the personal configuration folder of wireshark\\
	& 4. Open the pcap file for this test with wireshark\\
	& 5. Look at the different struct members that are displayed in wireshark\\
	\midrule
	Expected result  & 2. The utility should provide the user some text expressing the success of the LUA-file generation \\
	& 5. The unions should be displayed as subtrees with the names of all the union members and their values, even the ones that haven't been explicitly set\\
	\bottomrule
\end{tabularx}}
\end{table}

\begin{table}[ht] \footnotesize \center
\caption{Test case TID18}\label{tab:TID18}
\noindent\makebox[\textwidth]{%
\begin{tabularx}{1.2\textwidth}{l X} 
	\toprule
	Header & Description \\
	\midrule
	Description & Support filter and search in wireshark  \\
	Tester & Erik Bergersen \\
	Prerequisites & Wireshark has to be up and running with the trailer\_test.lua \\
	Feature & The dissector must support Wireshark's built-in filter and search on attributes \\
	\midrule
	\multirow{3}{*}{Execution} & 1. Open the pcap-file, which contains trailer\_test packets.   \\
	& 2. Type the following in the filter-textbox: ''luatructs.message == 66'' and click apply\\
	& 3. Look on the packet view. \\
	\midrule
	Expected result  & 3. Only trailer\_test packets will be visible in the packet view.\\
	\bottomrule
\end{tabularx}}
\end{table}

\begin{table}[ht] \footnotesize \center
\caption{Test case TID19}\label{tab:TID19}
\noindent\makebox[\textwidth]{%
\begin{tabularx}{1.2\textwidth}{l X}
 	\toprule
  	Header & Description \\
  	\midrule
  	Description &  Support WIN32, \_WIN64, \_sparc etc \\
  	Tester & Lars Solvoll Tønder \\
  	Prerequisites & The utility has to be up and running \\
  	Feature &  Test that the utility is able to support c-header files with platform definition preprocessor macros (e.g. \_WIN32, \_WIN64, \_sparc, etc.)\\
  	\midrule
  	\multirow{5}{*}{Execution} & 1. Feed the utility with c-header files with platform definition preprocessor macros (e.g. \_WIN32, \_WIN64, \_sparc, etc.). \\
 	& 2. Read output from the utility \\
 	& 3. Move the resulting dissector into the plugins folder in the personal configuration folder of wireshark\\
 	& 4. Open the pcap file for this test with wireshark\\
 	& 5. Look at the different struct members that are displayed in wireshark\\
	\midrule
	\multirow{2}{*}{Expected result}  & 2. The program should provide the user with some text expressing the success of the LUA-file generation. \\
	& 5. The different structs should be displayed as having the right struct members according to the c-header definitions. For example, the member defined surronded by \texttt{\#ifdef \_WIN32 ... \#endif} directives should be displayed correctly in WIN32 struct. \\
	\bottomrule
\end{tabularx}}
\end{table}

\begin{table}[ht] \footnotesize \center
\caption{Test case TID20}\label{tab:TID20}
\noindent\makebox[\textwidth]{%
\begin{tabularx}{1.2\textwidth}{l X}
	\toprule
	Header & Description \\
	\midrule
	Description & Supporting the use of flags specifying platforms to display member values correctly \\
	Tester & Lars Solvoll Tønder \\
	Prerequisites & A dissector for dissecting a header according to it's configuration, as well as a pcap file for that header needs to be in place\\
	Feature &  Test that the utility is able to create dissectors that support the use of different flags. These flags should specify the originating platform of the packet and be used to display the member values properly.\\
	\midrule
	\multirow{2}{*}{Execution} & 1. Open a pcap file which includes packets with different flags but the same structs and member values just with different sizes according to the originating platform specified in the flags\\
	& 2. Look at how the different packets are displayed in wireshark\\
	\midrule
	Expected result  & 2. The different packages should have the exact same member values just with different flag-values \\
	\bottomrule
\end{tabularx}}
\end{table}

\clearpage

\begin{table}[ht] \footnotesize \center
\caption{Test case TID21}\label{tab:TID21}
\noindent\makebox[\textwidth]{%
\begin{tabularx}{1.2\textwidth}{l X} 
	\toprule
	Header & Description \\
	\midrule
	Description & Supporting platforms with different endian  \\
	Tester & Erik Bergersen \\
	Prerequisites & There have to exist a header-file, config-file and pcap-file to do the test. \\
	Feature & Generate dissectors which support both little and big endian platforms \\
	\midrule
	\multirow{5}{*}{Execution} & 1. Feed the utility with the name of a c-header file with a struct \\
	& 2. Read the output \\
	& 3. Move the resulting dissector into the plugins folder in the personal configuration folder of wireshark\\
	& 4. Open the pcap file for this test with wireshark\\
	& 5. Look at the different members in wireshark that have data types that can be affected by endianness. \\
	\midrule
	\multirow{2}{*}{Expected result}  & 2. The program should provide the user with some text expressing the success of the LUA-file generation \\
	& 5. The packets sent from platforms with different endian, should display the same values.\\
	\bottomrule
\end{tabularx}}
\end{table}

\begin{table}[ht] \footnotesize \center
\caption{Test case TID22}\label{tab:TID22}
\noindent\makebox[\textwidth]{%
\begin{tabularx}{1.2\textwidth}{l X} 
	\toprule
	Header & Description \\
	\midrule
	Description & Supporting alignments  \\
	Tester & Erik Bergersen \\
	Prerequisites & There have to exist a header-file, config-file and pcap-file to do the test. \\
	Feature & Genereate dissectors on platforms with different alignment \\
	\midrule
	\multirow{5}{*}{Execution} & 1. Feed the utility with the name of a c-header file with a struct \\
	& 2. Read the output \\
	& 3. Move the resulting dissector into the plugins folder in the personal configuration folder of wireshark\\
	& 4. Open the pcap file for this test with wireshark\\
	& 5. Look at the different packets from platforms that uses different alignments. \\
	\midrule
	\multirow{2}{*}{Expected result}  & 2. The program should provide the user with some text expressing the success of the LUA-file generation \\
	& 5. The packets sent from platforms that uses different alignments, should display equal values.\\
	\bottomrule
\end{tabularx}}
\end{table}

\begin{table}[ht] \footnotesize \center
\caption{Test case TID23}\label{tab:TID23}
\noindent\makebox[\textwidth]{%
\begin{tabularx}{1.2\textwidth}{l X} 
	\toprule
	Header & Description \\
	\midrule
	Description & Handling Lua keywords  \\
	Tester & Erik Bergersen \\
	Prerequisites & There have to exist a header-file, config-file and pcap-file to do the test. \\
	Feature & The utility should support struct members, that have equal names as keywords in Lua \\
	\midrule
	\multirow{5}{*}{Execution} & 1. Feed the utility with the name of a c-header file with a struct \\
	& 2. Read the output \\
	& 3. Move the resulting dissector into the plugins folder in the personal configuration folder of wireshark\\
	& 4. Open the pcap file for this test with wireshark\\
	& 5. Check that the dissector work as it should. \\
	\midrule
	\multirow{2}{*}{Expected result}  & 2. The program should provide the user with some text expressing the success of the LUA-file generation \\
	& 5. Wireshark should not display an error message for invalid Lua code.\\
	\bottomrule
\end{tabularx}}
\end{table}

\begin{table}[ht] \footnotesize \center
\caption{Test case TID24}\label{tab:TID24}
\noindent\makebox[\textwidth]{%
\begin{tabularx}{1.2\textwidth}{l X}
	\toprule
	Header & Description \\
	\midrule
	Description & Sprint 3 functionality test \\
	Tester & Lars Solvoll Tønder \\
	Prerequisites & The utility has to have been installed on the system as well as the attest testing framework.The file black\_box.py must also be present in the test folder of Csjark \\
	Feature & Checking that the utility is able to create a valid dissector from header files with all of the data types that were to be supported for sprint 3\\
	\midrule
	\multirow{3}{*}{Execution} & 1. Navigate to the test folder inside the folder where CSjark is installed through the terminal or command line \\
	& 2.  type "python -m attest" into the terminal or command line and then press enter  \\
	& 3. Read the output\\
	\midrule
	Expected result  & 3. The user should be presented with some text expressing the failure of 0 assertions  \\
	\bottomrule
\end{tabularx}}
\end{table}

\begin{table}[ht] \footnotesize \center
\caption{Test case TID25}\label{tab:TID25}
\noindent\makebox[\textwidth]{%
\begin{tabularx}{1.2\textwidth}{l X}
	\toprule
	Header & Description \\
	\midrule
	Description & Sprint 2 functionality test \\
	Tester & Lars Solvoll Tønder \\
	Prerequisites & The utility has to have been installed on the system as well as the attest testing framework. The file requirements.py must also be present in the test folder of csjark \\
	Feature & Checking that the utility is able to support all of the features required by the customer \\
	\midrule
	\multirow{3}{*}{Execution} & 1. Navigate to the folder where CSjark is installed through the terminal orcommand line \\
	& 2.  type "python -m attest" into the terminal or command line and then press enter  \\
	& 3. Read the output\\
	\midrule
	Expected result  & 3. The user should be presented with some text expressing the failure of 0 assertions  \\
	\bottomrule
\end{tabularx}}
\end{table}

%----------------------
\section{Sprint 4}
%----------------------

\begin{table}[ht] \footnotesize \center
\caption{Test case TID26}\label{tab:TID26}
\noindent\makebox[\textwidth]{%
\begin{tabularx}{1.2\textwidth}{l X}
	\toprule
	Header & Description \\
	\midrule
	Description & Including system-headers \\
	Tester & Lars Solvoll Tønder \\
	Prerequisites & The utility has to be installed on the system and there has to exist a pcap file which is associated with this test \\
	Feature & Checking that the utility is able to support headers which use system headers \\
	\midrule
	\multirow{5}{*}{Execution} & 1.Feed the utility with the name of a c-header file that includes system-headers and its configuration file \\
	& 2.  Read the output\\
	& 3. Copy the resulting dissectors into the plugins folder of the personal configuration in wireshark\\
	& 4. Run wireshark with the pcap file associated with this test\\
	& 5. Look at the resulting structs and members are displayed in wireshark\\
	\midrule
	\multirow{2}{*}{Expected result}
	& 2. The user should be presented with some text expressing the success of generating dissectors\\
	& 5. The structs and struct members defined in the system headers should be displayed as having a value and not just hex data\\
	\bottomrule
\end{tabularx}}
\end{table}

\begin{table}[ht] \footnotesize \center
\caption{Test case TID27}\label{tab:TID27}
\noindent\makebox[\textwidth]{%
\begin{tabularx}{1.2\textwidth}{l X}
	\toprule
	Header & Description \\
	\midrule
	Description & Ignoring \#pragma directives \\
	Tester & Lars Solvoll Tønder \\
	Prerequisites & The utility has to be installed on the system and there needs to exist a pcap file which is associated with this test \\
	Feature & Making sure that the utility is able to parse header files with the \#pragma directive by just ignoring that directive\\
	\midrule
	\multirow{2}{*}{Execution} & 1.Feed the utility with the name of a c-header file that contains a \#pragma directive and it's configuration file\\
	& 2.  Read the output\\
	\midrule
	\multirow{1}{*}{Expected result}
	& 2. The user should be presented with some text expressing the success of generating dissectors\\
	\bottomrule
\end{tabularx}}
\end{table}

\begin{table}[ht] \footnotesize \center
\caption{Test case TID28}\label{tab:TID28}
\noindent\makebox[\textwidth]{%
\begin{tabularx}{1.2\textwidth}{l X}
	\toprule
	Header & Description \\
	\midrule
	Description & Improve generated Lua output by removing platform prefix\\
	Tester & Lars Solvoll Tønder \\
	Prerequisites & The utility has to be installed on the system \\
	Feature & Making sure the utility only generates one dissector for the struct instead of several, but still keeping all of the functionality\\
	\midrule
	\multirow{7}{*}{Execution} & 1.Feed the utility with any c-header file and it's configration\\
	& 2.  Read the output\\
	& 3. Copy the resulting dissectors into the plugins folder of the personal configuration in wireshark\\
	& 4. Run wireshark\\
	& 5. Open the dissector tables menu entry from the Internals menu\\
	& 6. Click the luastructs tree entry\\
	& 7. Inspect its contents\\ 
	\midrule
	\multirow{2}{*}{Expected result}
	& 2. The user should be presented with some text expressing the success of generating dissectors\\
	& 7. There should only be one tree entry for each dissector, not one for each platform as well\\
\end{tabularx}}
\end{table}

\begin{table}[ht] \footnotesize \center
\caption{Test case TID29}\label{tab:TID29}
\noindent\makebox[\textwidth]{%
\begin{tabularx}{1.2\textwidth}{l X}
	\toprule
	Header & Description \\
	\midrule
	Description & Recursive searching of subfolders\\
	Tester & Lars Solvoll Tønder \\
	Prerequisites & The utility has to be installed on the system and there needs to exist a folder with folders that all have header files in them \\
	Feature & Checking that it is possible for the utility to be fed a folder with header files that has subfolders which are in turn inspected\\
	\midrule
	\multirow{3}{*}{Execution} & 1.Feed the utility with the name of a folder of header files which again has subfolders with other header files\\
	& 2.  Read the output\\
	& 3. Inspect the generated Lua files \\
	\midrule
	\multirow{2}{*}{Expected result}
	& 2. The user should be presented with some text expressing the success of generating dissectors for every header file in the subfolders\\
	& 3. There should be 1 file for each struct contained in the different header files located in the header folder and its subfolders \\
\end{tabularx}}
\end{table}

\begin{table}[ht] \footnotesize \center
\caption{Test case TID30}\label{tab:TID30}
\noindent\makebox[\textwidth]{%
\begin{tabularx}{1.2\textwidth}{l X}
	\toprule
	Header & Description \\
	\midrule
	Description & Finding include dependencies which are not explicitly set\\
	Tester & Lars Solvoll Tønder \\
	Prerequisites & The utility has to be installed on the system \\
	Feature & Check that the utility is able to identify include dependencies which are not explicitly set and use that information to parse files over again correctly\\
	\midrule
	\multirow{5}{*}{Execution} & 1.Feed the utility with a c-header file that has include dependencies which are not explicitly set, and it's configration file\\
	& 2.  Read the output\\
	& 3. Copy the resulting dissectors into the plugins folder of the personal configuration in wireshark\\
	& 4. Run wireshark with the pcap file associated with this test\\
	& 5. Inspect the different packets, their structs and member values\\
	\midrule
	\multirow{2}{*}{Expected result}
	& 2. The user should be presented with some text expressing the success of generating dissectors\\
	& 5. All structs and member values should be displayed as having proper values and not just hex data\\
\end{tabularx}}
\end{table}

\begin{table}[ht] \footnotesize \center
\caption{Test case TID31}\label{tab:TID31}
\noindent\makebox[\textwidth]{%
\begin{tabularx}{1.2\textwidth}{l X}
	\toprule
	Header & Description \\
	\midrule
	Description & Pointer support\\
	Tester & Lars Solvoll Tønder \\
	Prerequisites & The utility has to be installed on the system \\
	Feature & Checking that the utility is able to support the use of pointers in header files\\
	\midrule
	\multirow{5}{*}{Execution} & 1.Feed the utility with a c-header file that has pointers, and it's configration file\\
	& 2.  Read the output\\
	& 3. Copy the resulting dissectors into the plugins folder of the personal configuration in wireshark\\
	& 4. Run wireshark with the pcap file associated with this test\\
	& 5. Inspect the different packets, their structs and member values\\
	\midrule
	\multirow{2}{*}{Expected result}
	& 2. The user should be presented with some text expressing the success of generating dissectors\\
	& 5. All structs and member values should be displayed as having proper values and not just hex data\\
\end{tabularx}}
\end{table}

\begin{table}[ht] \footnotesize \center
\caption{Test case TID32}\label{tab:TID32}
\noindent\makebox[\textwidth]{%
\begin{tabularx}{1.2\textwidth}{l X}
	\toprule
	Header & Description \\
	\midrule
	Description & Enums in arrays\\
	Tester & Lars Solvoll Tønder \\
	Prerequisites & The utility has to be installed on the system \\
	Feature & Checking that the utility is able to support the use of enums inside of arrays\\
	\midrule
	\multirow{5}{*}{Execution} & 1.Feed the utility with a c-header file that has an array of enums, and it's configration file\\
	& 2.  Read the output\\
	& 3. Copy the resulting dissectors into the plugins folder of the personal configuration in wireshark\\
	& 4. Run wireshark with the pcap file associated with this test\\
	& 5. Inspect the different packets, their structs and member values\\
	\midrule
	\multirow{2}{*}{Expected result}
	& 2. The user should be presented with some text expressing the success of generating dissectors\\
	& 5. All structs that contain an array of enums should be displayed as containing enums instead of just integers\\
\end{tabularx}}
\end{table}

\begin{table}[ht] \footnotesize \center
\caption{Test case TID33}\label{tab:TID33}
\noindent\makebox[\textwidth]{%
\begin{tabularx}{1.2\textwidth}{l X}
	\toprule
	Header & Description \\
	\midrule
	Description & Supporting \#define as a command line argument\\
	Tester & Lars Solvoll Tønder \\
	Prerequisites & The utility has to be installed on the system \\
	Feature & Checking that it is possible to feed the utility with a \# define argument and have it force pre-processor to add a corresponding argument to the header files it is processing\\
	\midrule
	\multirow{5}{*}{Execution} & 1.Feed the utility with a c-header file, it's configration file and a \#define directive\\
	& 2.  Read the output\\
	& 3. Copy the resulting dissectors into the plugins folder of the personal configuration in wireshark\\
	& 4. Run wireshark with the pcap file associated with this test\\
	& 5. Inspect the different packets, their structs and member values\\
	\midrule
	\multirow{2}{*}{Expected result}
	& 2. The user should be presented with some text expressing the success of generating dissectors\\
	& 5. All structs and their member values should be displayed as having proper values and not just hex data\\
\end{tabularx}}
\end{table}

\begin{table}[ht] \footnotesize \center
\caption{Test case TID34}\label{tab:TID34}
\noindent\makebox[\textwidth]{%
\begin{tabularx}{1.2\textwidth}{l X}
	\toprule
	Header & Description \\
	\midrule
	Description & Multiple message ID's for one dissector\\
	Tester & Lars Solvoll Tønder \\
	Prerequisites & The utility has to be installed on the system \\
	Feature & Checking that the utility supports having more than one message ID per dissector\\
	\midrule
	\multirow{2}{*}{Execution} & 1.Feed the utility with a c-header file and it's configuration file which includes more than one message ID\\
	& 2.  Read the output\\ 
	\midrule
	\multirow{1}{*}{Expected result}
	& 2. The user should be presented with some text expressing the success of generating dissectors\\
\end{tabularx}}
\end{table}

\begin{table}[ht] \footnotesize \center
\caption{Test case TID35}\label{tab:TID35}
\noindent\makebox[\textwidth]{%
\begin{tabularx}{1.2\textwidth}{l X}
	\toprule
	Header & Description \\
	\midrule
	Description & Allowing configuration for unknown structs\\
	Tester & Lars Solvoll Tønder \\
	Prerequisites & The utility has to be installed on the system \\
	Feature & Checking that the utility supports being able to configure the size of unknown structs\\
	\midrule
	\multirow{5}{*}{Execution} & 1.Feed the utility with a c-header file that has unparseable members and it's configuration file which includes the size of the struct itself\\
	& 2.  Read the output\\
	& 3. Copy the resulting dissectors into the plugins folder of the personal configuration in wireshark\\
	& 4. Open wireshark with the pcap file associated with this test\\
	& 5. Inspect the different packets, their structs and member values\\ 
	\midrule
	\multirow{1}{*}{Expected result}
	& 2. The user should be presented with some text expressing the success of generating dissectors\\
	& 5. All of the members of each packets should have proper values except for the unparseable members which should only be displayed as containing hex data\\
\end{tabularx}}
\end{table}

\begin{table}[ht] \footnotesize \center
\caption{Test case TID36}\label{tab:TID36}
\noindent\makebox[\textwidth]{%
\begin{tabularx}{1.2\textwidth}{l X}
	\toprule
	Header & Description \\
	\midrule
	Description & Allowing configuration for unknown structs\\
	Tester & Lars Solvoll Tønder \\
	Prerequisites & The utility has to be installed on the system \\
	Feature & Checking that the utility supports being able to configure the size of unknown structs\\
	\midrule
	\multirow{5}{*}{Execution} & 1.Feed the utility with a c-header file that has unparseable members and it's configuration file which includes the size of the struct itself\\
	& 2.  Read the output\\
	& 3. Copy the resulting dissectors into the plugins folder of the personal configuration in wireshark\\
	& 4. Open wireshark with the pcap file associated with this test\\
	& 5. Inspect the different packets, their structs and member values\\ 
	\midrule
	\multirow{2}{*}{Expected result}
	& 2. The user should be presented with some text expressing the success of generating dissectors\\
	& 5. All of the members of each packets should have proper values except for the unparseable members which should only be displayed as containing hex data\\
\end{tabularx}}
\end{table}

\clearpage

\begin{table}[ht] \footnotesize \center
\caption{Test case TID37}\label{tab:TID37}
\noindent\makebox[\textwidth]{%
\begin{tabularx}{1.2\textwidth}{l X}
	\toprule
	Header & Description \\
	\midrule
	Description & Autogenerating configuration files for structs that has no config file of their own\\
	Tester & Lars Solvoll Tønder \\
	Prerequisites & The utility has to be installed on the system \\
	Feature & Checking that the utility is able to create template configuration files for all structs that does not currently have a configuration\\
	\midrule
	\multirow{3}{*}{Execution} & 1.Feed the utility with a c-header file which has several structs without any configuration files \\
	& 2.  Open the configuration folder where CSjark is installed\\
	& 3. Inspect the configuration files \\
	\midrule
	\multirow{1}{*}{Expected result}
	& 3. There should now be one configuration file present for each struct in the c-header file that has an empty template for filling in values for configuration\\
\end{tabularx}}
\end{table}

\begin{table}[ht] \footnotesize \center
\caption{Test case TID38}\label{tab:TID38}
\noindent\makebox[\textwidth]{%
\begin{tabularx}{1.2\textwidth}{l X}
	\toprule
	Header & Description \\
	\midrule
	Description & Only generating dissectors for structs with a valid ID\\
	Tester & Lars Solvoll Tønder \\
	Prerequisites & The utility has to be installed on the system \\
	Feature & Making sure that the utility only creates dissectors for files that have a valid ID specified in its configuration\\
	\midrule
	\multirow{2}{*}{Execution} & 1.Feed the utility with a c-header file and it's configuration file that has no valid ID\\
	& 2.  Read the output\\
	\midrule
	\multirow{1}{*}{Expected result}
	& 2. The user should be presented with a message saying that no dissectors were created and why\\
\end{tabularx}}
\end{table}

\begin{table}[ht] \footnotesize \center
\caption{Test case TID39}\label{tab:TID39}
\noindent\makebox[\textwidth]{%
\begin{tabularx}{1.2\textwidth}{l X}
	\toprule
	Header & Description \\
	\midrule
	Description & Guessing dissectors from packet size\\
	Tester & Lars Solvoll Tønder \\
	Prerequisites & The utility has to be installed on the system and wireshark has to have been loaded with a dissector for a struct of the same size as the one associated with this test \\
	Feature & Making sure the utility is able to guess which dissector to use based on packet size if there are no dissectors specified for the packet\\
	\midrule
	\multirow{2}{*}{Execution} & 1.Start wireshark with the pcap file associated with this test\\
	& 2.  Inspect the different packets, their structs and member values\\
	\midrule
	\multirow{1}{*}{Expected result}
	& 2. All of the packets should contain a struct with different members and values instead of just raw hex data\\
\end{tabularx}}
\end{table}

\begin{table}[ht] \footnotesize \center
\caption{Test case TID40}\label{tab:TID40}
\noindent\makebox[\textwidth]{%
\begin{tabularx}{1.2\textwidth}{l X}
	\toprule
	Header & Description \\
	\midrule
	Description & Invalid header\\
	Tester & Lars Solvoll Tønder \\
	Prerequisites & The utility has to be installed on the system\\
	Feature & Making sure the utility crashes if it receives an invalid header\\
	\midrule
	\multirow{2}{*}{Execution} & 1.Feed the utility with an invalid header file\\
	& 2.  Read the output\\
	\midrule
	\multirow{1}{*}{Expected result}
	& 2. The utility should crash and give an error message explaining why it crashed\\
\end{tabularx}}
\end{table}

\begin{table}[ht] \footnotesize \center
\caption{Test case TID41}\label{tab:TID41}
\noindent\makebox[\textwidth]{%
\begin{tabularx}{1.2\textwidth}{l X}
	\toprule
	Header & Description \\
	\midrule
	Description & Invalid header during batch mode\\
	Tester & Lars Solvoll Tønder \\
	Prerequisites & The utility has to be installed on the system\\
	Feature & Making sure the utility skips headers it is unable to parse during batch processing\\
	\midrule
	\multirow{2}{*}{Execution} & 1.Feed the utility with a folder containing several invalid header files\\
	& 2.  Read the output\\
	\midrule
	\multirow{1}{*}{Expected result}
	& 2. The utility should skip the files it is unable to parse and present the user with a message saying why it skipped the files it was unable to parse\\
\end{tabularx}}
\end{table}