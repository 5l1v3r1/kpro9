\documentclass[a4paper, 11pt]{report}
\usepackage[T1]{fontenc}
\usepackage[utf8]{inputenc}
\usepackage[english]{babel}
\usepackage{graphicx} % support graphics
\usepackage{hyperref} % links in the document
\usepackage{float} % position of figures
\usepackage{paralist} % inline lists

%\setcounter{tocdepth}{1} % Depth of table of contents

% Configure links in pdfs
\hypersetup{
    bookmarksopen=false, % Hide bookmarks menu
    colorlinks=true, % Don't wrap links in colored boxes
}

\title{Wireshark:\\ Automated generation of protocol dissectors}
\author{by Erik Bergersen, Sondre Johan Mannsverk,\\ Terje Snarby,
		Even Wiik Thomassen, Lars Solvoll Tønder,\\ Sigurd Wien
		and Jaroslav Fibichr}
\date{\today}

\begin{document}
\maketitle

\begin{abstract}
Wireshark is the world's foremost network protocol analyzer, and is the de
facto (and often de jure) standard across many industries and educational
institutions. Lua is a powerful, fast, lightweight, embeddable scripting
language, which can be used to extend Wireshark.

We intend to do something awesome with some software we are gonna write
which then will do something to make something easier.
\end{abstract}

\tableofcontents

\part{Some latex commands}
% This is a comment, it won't show up in the document

Here comes some important LaTeX commands and syntax.
This is an url: \url{http://www.wikibooks.org} and
\href{http://www.wikibooks.org}{this is the same url}.
Here comes a footnote\footnote{Yes, actually, it is a footnote}
This is \textbf{bold}, this is \emph{emphasis} and this is \textit{italics}.

This is a numbered list.
\begin{enumerate}
   \item The first item
   \item The second item
   \item The third etc \ldots
\end{enumerate}

\paragraph{Lorem irsum}
Lorem ipsum dolor sit amet, consectetur adipiscing elit. Donec a diam lectus.
Sed sit amet ipsum mauris. Maecenas congue ligula ac quam viverra nec
consectetur ante hendrerit. Donec et mollis dolor. Praesent et diam eget
libero egestas mattis sit amet vitae augue. Nam tincidunt congue enim, ut
porta lorem lacinia consectetur. Donec ut libero sed arcu vehicula ultricies a
non tortor. Lorem ipsum dolor sit amet, consectetur adipiscing elit.


\part{Project Plan}

\chapter{Project Description}
Project name: Wireshark: Automated generation of protocol dissectors
Project sponsor: Thales Norway AS

%\chapter{Use Cases}
%\section{Actors}
%\section{High-Level Use Case Diagram}
%\section{Low-Level Use Case Diagrams}

\chapter{Overview}
An overview over both functional and non-functional requirements

\chapter{Prioritization}

\chapter{Functional Requirements}

\chapter{Non-Functional Requirements}

\chapter{Test plan?}




\end{document}
