%=====================================
\chapter{Initial List of Requirements}
%=====================================
\label{app:initreqs}
The customer provided an initial requirements specification for the utility at
the start of the project, which can be seen in \autoref{sec:customerreq}.

We made some initial changes to the format, created some non-functional
requirements and added priority and complexity to each requirement.
This resulted in the initial function requirements listed in
\autoref{tab:req:init:funcreq} and initial non-functional requirements listed
in \autoref{tab:req:init:nonfuncreq}.
These changes were approved by the customer before the start of the first
sprint.


%-----------------------------------
\section{Requirements from Customer}
%-----------------------------------
\label{sec:customerreq}
The customer provided the following list of requirements, for the utility
we should create, at the start of the project.
\begin{description}
	\item[F01] The utility shall be able to read basic C language struct
		definitions, and generate a Wireshark dissector for the binary
		representation of the structs.
	\item[F02] The utility shall support structs with any of the basic data
		types (e.g. int, boolean, float, char) and structs.
	\item[F03] The utility shall be able to follow \#include <...> statements.
		This allows parsing structs that depend on structs or defines from
		other header files.
	\item[F04] Each struct may be connected to one or more references (integer value).
		For instance, a member parameter 'type' can have names for a set of values.
	\item[F05] The dissector shall be able to recognize invalid values for a
		struct member. Allowed ranges should be specified by configuration. An
		example is an integer that indictates a percentage between 0 and 100.
	\item[F06] A struct may have a header and/or trailer (other registered
		protocol). This must be configurable.
	\item[F07] The dissector shall be able to display each struct member.
		Structs within structs shall also be dissected and displayed.
	\item[F08] It shall be possible to configure special handling of specific
		data types. E.g. a 'time\_t' may be interpreted to contain a unixtime
		value, and be displayed as a date.
	\item[F09] An integer member may indicate that a variable number of other
		structs (array of structs) are following the current struct.
	\item[F10] Integers may be an enumerated named value or a bit string.
	\item[F11] The dissectors produced shall be able to handle binary input
		from at least Windows 32bit and 64bit, Solaris 64bit and Sparc.
		Example: BOOL is 1 byte on Solaris and 4 bytes on Win32. Endian and
		alignment also differs between the architectures.
\end{description}


%-----------------------------
\section{Initial Requirements}
%-----------------------------
Initial function requirements are listed in
\autoref{tab:req:init:funcreq} and initial non-functional requirements are
listed in \autoref{tab:req:init:nonfuncreq}.
\begin{table}[htbp] \footnotesize \center
\caption{Initial Functional Requirements\label{tab:req:init:funcreq}}
\noindent\makebox[\textwidth]{%
\begin{tabularx}{1.2\textwidth}{l X c c}
	\toprule
	ID & Description & Pri. & Cmp. \\
	\midrule
	FR1 & The utility must be able to read basic C language
		struct definitions from C header files.
		& H & \\
	FR1-A & The utility must support the following basic data types:
		int, float, char and boolean.
		& H & L \\
	FR1-B & The utility must support members of type enums.
		& H & L \\
	FR1-C & The utility must support members of type structs.
		& H & M \\
	FR1-D & The utility must support members of type unions.
		& M & M \\
	FR1-E & The utility must support member of type array.
		& H & M \\
	\midrule
	FR2 & The utility must be able to generate lua-script for Wireshark
		dissectors for the binary representation of C struct.
		& H & \\
	FR2-A & The dissector shall be able to display simple structs.
		& H & L \\
	FR2-B & The dissector shall be able to support structs within
		structs.
		& M & M \\
	FR2-C & The dissector must support Wireshark's built-in filter and
		search on attributes.
		& H & L \\
	\midrule
	FR3 & The utility must support C preprocessor directives and macros.
		& H & \\
	FR3-A & The utility shall support \#include.
		& H & L \\
	FR3-B & The utility shall support \#define and \#if.
		& H & L \\
	FR3-C & The utility shall support , \verb+_WIN32+,
		\verb+_WIN64+, \verb+__sparc__+, \verb+__sparc+ and \verb+sun+.
		& M & H \\
	\midrule
	FR4 & The utility must support user configuration.
		& M & \\
	FR4-A & The dissector shall be able to recognize invalid values for
		a struct member. Allowed ranges should be specified by configuration.
		& L & L \\
	FR4-B & Configuration must support integer members which represent
		enumerated named value or a bit string.
		& M & L \\
	FR4-C & Configuration must support custom handling of specific data
		types. E.g. a 'time\_t' may be interpreted to contain a unixtime value,
		and be displayed as a date.
		& L & M \\
	\midrule
	FR5 & A struct may have a header and/or trailer (other registered
		protocol). The configuration must support the use of integer members to
		indicate the number of other structs that will follow in the trailer
		& L & H \\
	\midrule
	FR6 & The dissectors must be able to handle binary input which size
		and endian depends on originating platform.
		& M & \\
	FR6-A & Flags must be specified for each platform.
		& M & M \\
	FR6-B & Flags within message headers should signal the platform.
		& M & H \\
	\midrule
	FR7 & The utility shall support parameters from command line.
		& H & \\
	FR7-A & Command line shall support parameters for C header file.
		& H & L \\
	FR7-B & Command line shall support for configuration file.
		& H & L \\
	FR7-C & Command line shall support batch mode of C header and
		configuration file.
		& L & M \\
	FR7-D & When running batch mode, dissectors that already are
		generated, shall not be regenerated, if the source are not modified
		since last run.
		& L & M \\
	\bottomrule
\end{tabularx}}
\end{table}

\begin{table}[htbp] \footnotesize \center
\caption{Initial Non-Functional Requirements\label{tab:req:init:nonfuncreq}}
\noindent\makebox[\textwidth]{%
\begin{tabularx}{1.2\textwidth}{l X c c}
	\toprule
	ID & Description & Pri. & Cmp. \\
	\midrule
	NR1 & The utility shall be able to run on latest Windows and Solaris
		operating system.
		& M & L \\
	\addlinespace
	NR2 & The dissector shall be able to run on Windows x86, Windows x86-64,
		Solaris x86, Solaris x86-64 and Solaris SPARC.
		& M & M \\
	\addlinespace
	NR3 & The utilities user interface shall be command line.
		& H & L \\
	\addlinespace
	NR4 & The configuration shall have sufficient documentation to allow a
		person with no previous knowledge of the system to be able to use it
		to generate LUA-scripts after five hours of reading.
		& M & M \\
	\addlinespace
	NR5 & The configuration should have sufficient documentation to allow a
		person, already proficient with the system, to understand the code
		well enough to be able to extend it’s functionality after four hours of
		reading.
		& M & M \\
	\addlinespace
	NR6 & The utility code should follow standard python coding convention as
		specified by PEP8, and try to follow python style guidelines defined
		by PEP20.
		& H & L \\
	\addlinespace
	NR7 & The utilities code should be documented by python docstrings which
		should explain the use of the code. Python modules, classes, functions
		and methods should have docstrings.
		& M & L \\
	\bottomrule
\end{tabularx}}
\end{table}

