%=================
\chapter{Sprint 4}
%=================
TODO

%------------------------
\section{Sprint Planning}
%------------------------
The fourth sprint will be the last iteration in this project. Testing and bugfixing will take most of the sprint work hours, because it is very important to make the utility work properly on Thales source code. Now the utility fail on most of the structs: Stig tried to generate dissectors for 580 structs, where only 4 of them succeeded. It will probably just be some small tweaks needed, to make it work for most of the structs.

We know that some of our earlier implementation do not work as intended or is incomplete. The customer has given us feedback on the fulfilled requirements, and pointed out what needs to be improved. These are important work items of this sprint backlog \ref{sp4:backlog}. In addition we got new requirements that would be nice if we implemented, some which will be convenient for the customer to have. For the optional requirements we do not have time to implement, we have agreed to write a specification on how to implement them. 

There will not be another sprint after this one, so we can not postpone any work items. Because of this, we had to make some of the new requirements from the customer optional, and also telling the customer that we will probably not be able complete all their desires. We are tyring to improve from tha last sprint, are not going to add more tasks than we can complete this sprint. This sprint we plan to use close to 350 hours, but it is still likely that we will use more and are prepared for that. It will certainly pop up new tasks and unforeseen bugs that we have to fix.

\subsection{Duration}
%-----------------------
The sprint started with the planning meeting the 2nd of November and our work started the following day. The sprint duration is 14 days, and will end the 15th of November with a review meeting. 

\subsection{Sprint Goal}
%-----------------------
The goal of this sprint will be focusing on fixing and implement function that the customer will need to use the utility on their source code. The most important thing to focus on in beginning of the sprint will be to implement support for \#pragma directives and support for including header-files that are not included by the pre-processor. This is important, since it will make it possible for the customer to test the utility.

Since the deadline of the project is 24th November, there will also be a focus on preparing a presentation and improve the report for a final delivery. The team are also going to hold a presentation for the customers developers on the 17th November, and it is important to also focus on this presentation.

\subsection{Back Log}
%--------------------
This section contains the sprint backlog \ref{tab:sprint4req} and the timetable for the sprint Table \ref{tab:sprint4time}.  

\begin{table}[!htb] \small \center
\caption{Sprint 4 Requirement Work Items \label{tab:sprint4req}}
\begin{tabularx}{\textwidth}{l X c c}
	\toprule
	& & \multicolumn{2}{c}{Hours} \\
	\cmidrule(r){3-4}
	User story & Req. and Description & Est. & Act. \\
	\midrule
	\textbf{Impl.} &  & \textbf{27} & \textbf{-} \\
	\hyperref[tab:req:stories7]{US29} & FR3-A mod: support \#include of system headers &  8  & - \\
	\hyperref[tab:req:stories8]{US31} & FR3-D mod: ignore \#pragma directives & 2 & - \\
	\hyperref[tab:req:stories8]{US32} & FR3-E: Find include dependencies which are not explicit set & 16  & - \\
	\hyperref[tab:req:stories8]{US32} & FR4-H: Autogenerate configuration files for structs with no config file & 1  & - \\
	\addlinespace
	\textbf{Fixes} &  & \textbf{16} & \textbf{-} \\
	\hyperref[tab:req:stories7]{US29} & Fix1: Improve generated Lua output - Remove platform prefix & 10 & - \\
	\hyperref[tab:req:stories7]{US29} & Fix2: Fix bugs to be able to process customers files & 12 & - \\
	\hyperref[tab:req:stories7]{US29} & Fix3: Array bug in text & 2 & - \\
	\hyperref[tab:req:stories7]{US29} & Fix4: Pointer support (array) & 1 & - \\
	\hyperref[tab:req:stories7]{US29} & Fix5: Enum in arrays & 1 & - \\		
	\hyperref[tab:req:stories7]{US29} & Fix6: Support command line arguments for Cpp macros & 1 & - \\
	\hyperref[tab:req:stories7]{US29} & Fix7: Multiple message ID's for one dissector & 2 & - \\
	\hyperref[tab:req:stories7]{US29} & Fix8: Allow configuration of size for unknown structs & 4 & - \\
	\hyperref[tab:req:stories7]{US29} & FR6-C: Batch mode fix (recursive searching for subfolders) &  8  & - \\
	\hyperref[tab:req:stories7]{US29} & FR6-C: Support command line arguments for Cpp "--Include" (folders) & 1 & - \\
	\addlinespace
	\textbf{Testing} &  & \textbf{19} & \textbf{-} \\
	 & FR4-B: Custom \Gls{lua} configuration & 2 & - \\
	 & FR5-C: \Glspl{dissector} support both little and big \gls{endian} & 1 & - \\
	 & FR5-D: \Glspl{dissector} support different sizes from flags & 2 & - \\
	 & FR3-C: Support WIN32, WIN64, \gls{asparc} etc & 4 & - \\
	 & FR5-B: \Glspl{dissector} support memory alignement & 8 & - \\
	 & FR1-D: Support members of type \gls{union} & 1 & - \\
	 & FR2: Handle \Gls{lua} reserved definition names & 1 & - \\
	\addlinespace
	\textbf{Doc.} &  & \textbf{11} & \textbf{-} \\
	 & FR4-B: Custom \Gls{lua} configuration & 2 & - \\
	\hyperref[tab:req:stories8]{US34} & FR4-C: Support custom handling of specific data types & 2 & - \\
	\hyperref[tab:req:stories9]{US36} & FR5: User documentation for what platform that the \gls{utility} support & 3 & - \\
	\hyperref[tab:req:stories9]{US37} & FR5: Create developer manual from python docstrings (autodoc plugin) & 4 & - \\
	\midrule
	& Total: & 102 &  -\\
	\bottomrule
\end{tabularx}
\end{table}


\begin{table}[!htb] \small \center
\caption{Sprint 4 Timetable\label{tab:sprint4time}}
\begin{tabularx}{\textwidth}{X c c}
	\toprule
	& \multicolumn{2}{c}{Hours} \\
	\cmidrule(r){2-3}
	Description & Est. & Act. \\
	\midrule
	\textbf{Sprint planning} & \textbf{30} & \textbf{47.5} \\
	\addlinespace
	\textbf{Sprint 3 requirements} & \textbf{102} & \textbf{-} \\
	Implementation & 56 & - \\
	Testing & 19 & - \\
	User Documentation & 11 & - \\
	Fixes & 16 & - \\
	\addlinespace
	\textbf{Sprint review} & \textbf{20} & \textbf{-} \\
	\addlinespace
	\textbf{Sprint documentation} & \textbf{75} & \textbf{-} \\
	Sprint 1 document & 10 & -\\
	Sprint 2 document & 14 & - \\
	Sprint 3 document & 51 & - \\
	\addlinespace
	\textbf{Report work} & \textbf{42} & \textbf{-} \\
	User stories to LaTeX & 3 & 4\\
	Architecture update & 8 & -\\
	Glossaries and acronyms & 16 & -\\
	Requirement review & 15 & -\\
	Layout and correction & 15 & -\\
	\addlinespace
	\textbf{Lectures} & \textbf{21} & \textbf{14} \\
	\addlinespace
	\textbf{Meetings} & \textbf{57} & \textbf{-} \\
	Advisor meetings & 28 & - \\
	Customer meetings & 8 & - \\
	Stand-up meetings & 21 & - \\
	\textbf{Project management} & \textbf{20} & \textbf{-} \\
	\midrule
	Total: & 367 & - \\
	\bottomrule
\end{tabularx}
\end{table}



%----------------------
\section{System Design}
%----------------------


%-----------------------
\section{Implementation}
%-----------------------


%-----------------------
\section{Sprint Testing}
%-----------------------


%--------------------------
\section{Customer Feedback}
%--------------------------


%--------------------------
\section{Sprint Evaluation}
%--------------------------


