\section{Sprint 1 Tests}
\begin{table}[H]
\begin{tabularx}{\textwidth}{l X}
\hline\hline
Portion & Description\\[0.5ex]
\hline
Test identifier & TID01 Supporting parameters for c-header file\\[0.5ex]
Tester & Team member responsible for the test\\[0.5ex]
Prerequisites & The program has to have been compiled on the system\\[0.5ex]
Feature & Test that we are able to feed the solution with a c-header file and have it get dissected\\
Execution & 1.Start the program\newline
		2.Write the name of the c-header file in the command line\newline
		 3.Read the output given by the program\\ 
Expected result & 1.The program should start up without any errors and present the user with a command line interface\newline
3. The user should be presented with some text expressing the success of the LUA-file generation\\[0.5ex]
\hline\hline
\end{tabularx}


\begin{tabularx}{\textwidth}{l X}
\hline\hline
Portion & Description\\[0.5ex]
\hline
Test identifier & TID02 Supporting basic data types\\[0.5ex]
Tester & Team member responsible for the test\\[0.5ex]
Prerequisites & The program has to have been started\\[0.5ex]
Feature & Test that our utility will be able to make a dissectors for C-header files including the following basic data types: int, float, char and boolean \\
Execution & 1. Write the name of a c-header file which includes the aforementioned basic data types.\newline
		2. Read the output given by the program\\ 
Expected result & 2.The program should provide the user with some text expressing the success of the LUA-file generation\\[0.5ex]
\hline\hline
\end{tabularx}


\begin{tabularx}{\textwidth}{l X}
\hline\hline
Portion & Description\\[0.5ex]
\hline
Test identifier & TID03 Displaying simple structs\\[0.5ex]
Tester & Team member responsible for the test\\[0.5ex]
Prerequisites & The utility has already made a dissector\\[0.5ex]
Feature & Test that our utility is able to generate dissectors that displays simple structs \\
Execution & 1. Open wireshark and run the dissector.\newline
		2. Run the dissector on some captured data of simple structs
		3. Read the output \\
Expected result & 1. Wireshark should be able to load the dissector without any errors \newline
			3. Wireshark should display the data inside the structs sent in the capture data \\[0.5ex]
\hline\hline
\end{tabularx}


\begin{tabularx}{\textwidth}{l X}
\hline\hline
Portion & Description\\[0.5ex]
\hline
Test identifier & TID04 supporting \#include\\[0.5ex]
Tester & Team member responsible for the test\\[0.5ex]
Prerequisites & The utility has to be up and running\\[0.5ex]
Feature & Test that our utility supports c-header files with the \#include directive \\
Execution & 1. Write the name of a C-header file with an \#include directive
		2. Read the output\\
Expected result & 2.The program should provide the user with some text expressing the success of the LUA-file\\ generation\\[0.5ex]
\hline\hline
\end{tabularx}

\begin{tabularx}{\textwidth}{l X}
\hline\hline
Portion & Description\\[0.5ex]
\hline
Test identifier & TID05 supporting \#define and \#if\\[0.5ex]
Tester & Team member responsible for the test\\[0.5ex]
Prerequisites & The utility has to be up and running\\[0.5ex]
Feature & Test that our utility supports c-header files with \#define and \#if directives \\
Execution & 1. Write the name of a C-header file with a \#define and \#if directives
		2. Read the output\\
Expected result & 2.The program should provide the user with some text expressing the success of the LUA-file generation\\[0.5ex]
\hline\hline
\end{tabularx}

\begin{tabularx}{\textwidth}{l X}
\hline\hline
Portion & Description\\[0.5ex]
\hline
Test identifier & TID06 supporting configuration files\\[0.5ex]
Tester & Team member responsible for the test\\[0.5ex]
Prerequisites & The utility has to be up and running\\[0.5ex]
Feature & Test that our utility supports reading data from a configuration file \\[0.5ex]
Execution & PLACEHOLDER FOR NOW\\[0.5ex]
Expected result & PLACEHOLDER FOR NOW\\[0.5ex]
\hline\hline
\end{tabularx}

\end{table}

