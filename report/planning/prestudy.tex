%============================
\chapter{Preliminary Studies}
%============================
This chapter presents the preliminary studies for this project.
In \autoref{sec:pre:similar} we have examined existing solutions, in
\autoref{sec:pre:method} we provide a description of two popular software
development methodology, in \autoref{sec:pre:langs} we describe possible
languages and technology necessary for developing the utility, and in
\autoref{sec:pre:eval} we provide the justifications for the choices we have
made. In \autoref{sec:pre:framework} we provide a description of the framework
our utility will require. At the end of the chapter, in
\autoref{sec:pre:license}, we describe the license of our utility.

\section{Similar Solutions}
%--------------------------
\label{sec:pre:similar}
We started by searching for existing solutions in the problem space. This
search turned up idl2wrs\footnote{\url{http://wiki.wireshark.org/idl2wrs}}.
The other solution was suggested by our customer,
Asn2wrs\footnote{\url{http://wiki.wireshark.org/Asn2wrs}}.
Both of these solutions are bundled with Wireshark.

\subsection{idl2wrs}
A tool for generating Wireshark dissectors from IDL files. The tool is written
in python, and generates dissectors in C from IDL specifications. Interface
Description Language (IDL) is used to interface in a language-neutral way,
which enables communication between software of different languages. It is
used for example in Sun RPC and CORBA. Since idl2wrs takes input in a different
language than us, and creates dissectors in a different language than us, we
can not reuse any of its code. Instead we will look at its architecture and
data structures, especially how it generates dissectors.

\subsection{asn2wrs}
A tool for generating Wireshark dissectors from ASN.1 protocols. It requires
four input files: an ASN-1 protocol description, a configuration file and two
template files. Advantages of using Asn2wrs are faster development from
easier recompilation, and plugins which are easy to distribute. Disadvantages
are that code and makefiles are more complex. Our customer can't use this
solution as it would require them to rewrite their C structs to ASN.1
descriptions, which would take very long time.

\section{Software Development Methodology}
%-----------------------------------------
\label{sec:pre:method}

\subsection{Waterfall}
\label{sec:pre:waterfall}
Waterfall is a software development methodology based on sequential phases.
It consist of the following phases: requirement specification, design,
implementation, integration, testing, deployment, and maintenance. In its pure
form, these phases are non overlapping and one way only, which means that each
phase must be fully completed before the next can begin.

\subsubsection{Requirement specification}
This phase is about receiving requirements from a customer and then formalising
these into concrete functional and non-functional requirements. These will
again be further broken down into smaller work items that are easy to quantify
in terms of time of use and importance. These metrics may help distinguish
which features are to be prioritised.

\subsubsection{Design}
The design phase is about planning how to implement the features from the
requirement phase. The goal is to make a precise software architecture for the
project that dictates most of the implementation phase. This may include (but
not limited to) making class diagrams, data flow diagrams, state machines, user
interface mock-ups, etc.

\subsubsection{Implementation}
This phase is about implementing/coding the design made in the design phase on
a component level.

\subsubsection{Integration}
This phase is about integrating the different components that results from the
implementation phase.

\subsubsection{Testing}
The testing phase aims to thoroughly test the result of the implementation and
integration phase. The goal is to find and fix bugs introduced in both these
phases.

\subsubsection{Deployment}
The deployment phase is about delivering the resulting software to the
customer. This may include installing the software on their systems. This is
also the phase where the customer either accepts or rejects the resulting
software.

\subsubsection{Maintenance}
Large software projects are almost impossible to get completely bug free, and
therefore a certain amount of maintenance may be required. The obvious tasks
are to either fix or provide viable workarounds for problems that appear during
normal use. Maintenance may also include developing new features that the
customer finds the need for.

\subsection{Scrum}
Scrum is an agile development methodology based on the philosophy that it is
impossible to completely and accurately plan everything in a software project
before you begin. It is therefore more or less based on iterations of the
waterfall phases described in \autoref{sec:pre:waterfall}, but instead of
having these phases being strictly sequential, they are run in a more
'as needed' basis. Each iteration in scrum is called a sprint and is typically
lasting from two and up to four weeks. This time period is fixed for each
project, so the sprint will always end on time. To make this possible, features
that are not completed on time is deferred to a later sprint. Each sprint
should result in a runnable product that potentially could deliver some value
to the customer, even if this requires some redundant work.

\subsubsection{Main scrum roles}
\begin{description}
	\item[Scrum master] has the responsibility of maintaining the process and
		for removing obstacles for other team members. In short, the scrum
		master tries to keep the other team members focused on their tasks.
	\item[Product owner] represents and speaks for the customer. Not
		necessarily a part of the customers organization, but must have the
		stated authorities.
	\item[Team members] responsible for making and delivering the product.
		Should consist of a self organizing team of five to nine persons with
		cross functional skill set.
\end{description}

\subsubsection{Scrum artifacts}
\begin{description}
	\item[Product backlog] contains a high level description of all the desired
		features for the project. These should be prioritised based on their
		business value and evolve along with the project.
	\item[Sprint backlog] contains what the team is committed to complete over
		the next sprint. These commitments are features broken down into work
		items. These items should not be larger than 16 hours of work, and they
		should be described so that everyone in the team could contribute to
		implementing them.
	\item[Burn down chart] A daily updated chart consisting of what work
		remains in the sprint. Its purpose is both to show what work to do next
		and to give a visual representation of the work progress.
\end{description}

A sprint begins with the sprint planning meeting, which consists of two
stages. In the first, the team and the product owner prioritizes the product
backlog. In the second, the team discusses what features they can commit to
(based on priority) and brake these down into work items, which are added to
the sprint backlog. This should include giving each item an estimated
completion time.

The sprint itself consists of producing what is required for completing work
items, updating the burn down chart, and daily scrum meetings. In these daily
meetings each team member provides a short update of what they did the day
before, what they plan to do today and what problems might be in their way.
These problems should not be discussed in this meeting, but rather dealt with
separately after the meeting (this is the scrum masters responsibility).

At the end of the sprint cycle, the team should hold a scrum review meeting.
This meeting is about discussing what was completed and what was not, and to
demonstrate the completed features for the customer.

After the review meeting, a separate retrospect meeting should be held with all
the team members where all members shares their reflections of how the sprint
went and how to improve the next sprint. This is important for improving the
process.

\section{Development Languages and Technologies}
%-----------------------------------------------
\label{sec:pre:langs}

We estimate that the most challenging part of the utility will be parsing of C
header files. There are three distinct paths for solving it.
\begin{enumerate}
	\item We can build our own custom parser.
	\item We can use one or more libraries which does the parsing for us.
	\item We can use a complete tool chain like GCC or Clang.
\end{enumerate}

\noindent Each of these options has their advantages and drawbacks. While a
custom parser would give us complete control over each part of the parsing, it
would be a tremendous undertaking depending of how much of the C language
specification and C preprocessor directives we have to support. On the other
hand using a tool chain would add a large dependency to our solution, and is
non-trivial to install and use on the Windows platform.

We intend to investigate these possible solutions in Java first, and if
necessarily also investigate other programming languages if Java is not suited.

\subsection{Java}
Java is an object-oriented, structured, imperative, statically typed
programming language. It was originally developed by Sun Microsystems, which
is now a subsidiary of Oracle Corporation. Java was released in 1995, and it
derived much of its syntax from C and C++, but with fewer low-level facilities.
Java’s strength are portability, automatic memory management, security, good
documentation and an extensive standard library. Java has several tools and
libraries of varying quality for creating parsers, like ANTLR and JAVACC.

\subsubsection{ANTLR}
ANother Tool for Language Recognition (ANTLR) is a compiler compiler tool for
creating lexers and parsers from grammar files. It can create these compilers
for several different target languages, including Java and Python. There exists
ANTLR grammar files for C, C preprocessor and ASN.1. These configure ANTLR to
create Java lexers and parsers which reads and validates inputted source code
files.

\subsection{Python}
Python is a general-purpose, multi-paradigm, object-oriented, imperative,
dynamically typed programming language. It was created by Guido van Rossum,
and is today developed by Python Software Foundation and the Python community.
Python’s strength include automatic memory management, large and comprehensive
standard library, portability, powerful but very clear, concise and simple
syntax. There exists several pure Python libraries for creating lexers and
parsers, like PLY, pycparser and cppheaderparser.

\subsubsection{PLY}
PLY is a python alternative to the popular lexer and parser compilers lex and
yacc. It also comes with a 95\% completed C preprocessor in case we are
required to modify the preprocessor for our utility. Other special purpose
parsers like pycparser and cppheaderparser use PLY.

\subsubsection{pycparser}
There are two python C parsers with the same name, pycparser (from google code)
and PyCParser (from github). While they both aim to solve the same problem, the
first one appears to have better documentation, be a more mature project and
support more of the C99 specification. pycparser requires PLY to work.

\subsubsection{cppheaderparser}
Cppheaderparser is a parser for C++ header files written in python. It is an
alternative for pycparser in case we need to parse C++ files instead of simply
C files. It depends on PLY.

\subsubsection{Testing frameworks}
Python has several frameworks for unit testing, we decided to look at nose,
py.test and Attest to find which would best suit our needs. They are somewhat
similar, being modern python testing frameworks, but they differ in maturity
and what is often called magic in the python community. py.test is the most
mature but also the most magic, it uses a lot of introspection to discover
tests and it has no API. nose is heavily influenced by py.test, but it tries to
be more explicit, and provides an API. Attest is the youngest testing
framework, and it like nose has little magic but instead focuses on providing
a very pythonic API. Being the youngest also means it has the least
documentation, functionality and plugins. Therefore Attest might be the easiest
testing framework to learn. Independent of testing framework, we can use
coverage.py for testing how well our unit tests cover our code.

\subsubsection{Sphinx}
Sphinx is a python tool for writing documentation, that makes it easy to create
intelligent and beautiful documentation. It is used for the standard python
documentation, and it is  popular in the python community. Sphinx uses
reStructuredText as its markup language, which is a easy-to-read,
what-you-see-is-what-you-get plaintext markup syntax and parser system. Our use
case for sphinx is writing documentation for our utility, how to use it and
configure it. We might also use readthedocs.org for free hosting of
documentation, and which support automatic pulling and compiling of
documentation from our github repository whenever we push any changes.

\subsection{GCC}
GNU Compiler Collection (GCC) is a compiler system which has front ends which
parse C and C++ code, written in C and C++. It can be used in our utility as
an external tool which does the parsing and outputs an intermediate language
representation which we can parse/search to find the C struct definitions. Its
drawbacks are a lack of flexibility if we need to change its behaviour, and we
will still need to write a custom parser or use something like GCC-XML and an
XML parser. 

\subsection{Clang}
Clang is a compiler front end for C, C++, Objective-C and Objective-C++,
written in C++. Clang differ from GCC as it behaves as a library rather than an
external tool, but for Java we will have to use it like GCC because there
aren’t any Clang Java bindings. It supports outputting the abstract syntax
tree as XML, which our utility then will need to parse. Clang provides bindings
for Python so there it can be used as a library, but its drawback is a lack of
flexibility, much like GCC.

\subsection{Wireshark}
Wireshark is a free, open source network protocol analyzer. It lets you capture
and browse traffic running through a computer network. Wireshark is currently
being developed by the Wireshark team, a group of networking experts spanning
the globe. Because of its rich set of features and ease of use, Wireshark is de
facto standard in many different industries and educational institutions.
Wireshark is able to display, dissect and display data from a plethora of
different protocols. One of its strengths lies in the ease of which developers
can add their own dissectors, post-dissectors and taps. Dissectors can be
written in either C or LUA. Most dissectors are written in C for increased
speed. Lua-scripts are mostly used as prototypes or to process non time crucial
data as they don’t need compilation to be used. Our customer uses Wireshark not
only to browse through and filter regular networking traffic, but also for
monitoring inter-process communication where it is important to have a tool
which can easily be extended to dissect and display structures and data types
unique to the organization.

\subsection{Lua}
Lua is a multi-paradigm, dynamically typed programming language which is
designed to be lightweight so it can easily be embedded into applications.
Lua has only a few basic data structures: boolean, numbers, strings and table.
But Lua still implements advanced features such as first-class functions,
garbage collection, closures, coroutines and dynamic module loading. Lua was
created in 1993 at the Pontifical Catholic University of Rio de Janeiro, in
Brazil.

The output of our utility will be Wireshark dissectors written in Lua. While
Wireshark supports dissectors written in both C and Lua, Lua is preferred
because they can be added without recompiling Wireshark. Lua dissectors
interface with Wireshark through a simple API.

\subsection{Configuration}
Our utility needs a flexible configuration, simply because some of the
information we need to display does not exist in the C header files. For
example there exist no clear relationship between enumerated values in messages
and their names. These must be provided through a configuration. The
configuration will also need to support:
\begin{itemize}
	\item Defining data constraints, so the dissector can recognize invalid
		values for a struct member.
	\item Defining custom handling of special data types.
	\item  And possibly allow custom Lua dissectors to be defined and used in
		certain cases.
\end{itemize}

\subsubsection{ASN.1}
Abstract Syntax Notation One (ASN.1) is a notation for describing data
structures, the standard is defined by ISO 8824. ASN.1 is used for
representing, encoding, transferring and decoding of data. It provides a
fundamental tool to be used in application, which make it possible to exchange
information in a independent way.

Our customer uses ASN.1 for configuration of other utilities, and has therefore
suggested we investigate it for our utility. There exist two python libraries
for parsing of ASN.1

\subsubsection{YAML}
YAML (YAML Ain't Markup Language) is a data serialization format. It's designed
to be easy to read and write for humans. YAML syntax was designed to be easily
mapped to data types common to most high-level languages. While most
programming languages can use YAML for data serialization, YAML excels in
working with those languages that are fundamentally built around the three
basic primitives. These include the new wave of agile languages such as Perl,
Python, PHP, Ruby, and Javascript.

PyYAML is a parser for the python programming language, and it is available for
version 3.2 of python. The parser is licensed under MIT.

\subsubsection{configparser}
configparser is a module to manage user-editable configuration files for an
application. The configuration files are organized into sections, and each
section can contain name-value pairs for configuration data. Value
interpolation using Python formatting strings is also supported, to build
values that depend on one another.

configparser module is a standard part of Python, therefore there is no need
to additional installation and configuration.

\subsubsection{ConfigObj}
ConfigObj is a simple but powerful config file reader and writer (originally
based on ConfigParser). Its main feature is that it is very easy to use, with
a straightforward programmer's interface and a simple syntax for config files.
Among others, it has these additional features:
\begin{itemize}
	\item Nested sections (subsections), to any level
	\item List values
	\item Multiple line values
	\item String interpolation (substitution)
	\item Integrated with a powerful validation system
\end{itemize}

\noindent Currently, ConfigObj module exists only for Python up to version
2.7. It is under BSD license.

\subsection{Integrated Development Environment}

\subsubsection{PyCharm}
Pycharm is a cross platform, proprietary IDE for python. It has good support
for text editing, syntax highlighting, auto indentation, code navigation, code
completion and automatic error checking. There is also a decent debugger and
unit test support that can help finding errors and integrated version control
support (including git), which makes it easy to synchronize with a remote
repository. Most mentioned functions is also paired with keyboard shortcuts.

The downside with Pycharm is that it requires a relative expensive license. It
is, however, possible to apply for classroom licenses that are free of charge.
The latter is a requirement to make this IDE a viable option.

\subsubsection{Pyscripter}
Pyscripter is a Windows only, open source IDE for python. It has support for
basic text editing functions relevant to programming like syntax highlighting,
auto indentation, code completion, debugger and file management. It also has
some support for navigation the code, for example by offering to find the next
point in the code that references a certain variable or function. The mentioned
function mostly has keyboard shortcuts.

It does not have support for automatic error checking in the program, so it
won’t alert the user of spelling and syntax errors. It also lacks integration
with any version control systems like git or svn. The code completion and
code navigation is a little lacking. It will for example not suggest importing
files if you reference a class from another module and it cannot give a
complete list of usages of a function.

\subsubsection{Vim}
Vim is cross-platform, open source text editor originally created for the
Amiga. It is not regarded as an IDE, but it provides all the regular features
of text editors, including syntax highlighting, auto-completion,
auto-indentation, searching, multiple undo and redo etc. It can be configured
to support almost everything modern IDE’s support, and its extensive
customizability is considered parts of its strength. But it is also parts of
its weakness, it is very difficult for new vim users to learn how to use it
effectively. Therefore we do not suggest any team member which is not already
experienced with Vim to try it.

\section{Evaluation and Conclusion}
%----------------------------------
\label{sec:pre:eval}

\subsection{Development Process Choice}
We have chosen scrum for our development strategy. We do not have a lot of
experience with software development either individually or as a group, so we
have little personal knowledge of how much we are able to produce, and the
task may present challenges that we are not prepared for when the project
starts. For these reasons we believe that we need to take an agile approach to
this project. This way, we may both learn as we go, an adjust later iterations
by the result of the previous. We may also have something to deliver even if we
do not have time to implement all the desired features.

The scrum methodology fits these goals perfectly, and is therefore a natural
choice. The risk factor here is that all team members are mostly unfamiliar
with scrum, while we have at least a little knowledge about waterfall. We do,
however, think that the time and risk of learning will not be much compared to
the benefit it will give us over waterfall.

\subsection{Development Languages and Technologies Choice}
We originally selected Java as our programming language because it would run on
all the platforms required, it offered automatic memory management so it would
be easier to debug, and it was the only language everyone on the team had
experience with. We considered most strongly the second of the three outlined
possible solutions to parsing. We felt that writing our own C parser with C
preprocessor would possibly take up a lot of the available time of the project,
while the third option of using GCC or Clang would add a large dependency which
our customer would want to avoid. GCC and Clang is also not trivial to install
on Windows.

Using Java with ANTLR to generate C preprocessor, C lexer and C parser looked
very promising at first. Further inspections revealed that the C preprocessor
grammar was written in 2006, and had stopped working in 2008 as newer version
of ANTLR were not backwards compatible. The C grammar only provided validation
of the parsing, and did not create an abstract syntax tree, so it would require
some modification to get an output to work with. We did not test the ASN.1
grammar but comments inside it did not inspire confidence. It was clear that
this solution would require a lot of tinkering to get working, and ANTLR’s lack
of documentation was a huge risk for our project. This plus feedback from our
customer made us consider other programming languages for developing our
utility.

We looked at Python because our customer would prefer us to use it, and because
we had a Python-fanatic on our team. We found several Python libraries for
parsing of C files, and even one for parsing C++ header files.

We decided to use Python for this project because the parsing libraries for
Python came in working condition with decent documentation, and because we
were able to create a small working prototype in Python in just a few hours.
We estimate that it would take at least a week to achieve the same result in
Java.

We also decided to use python libraries PLY and pycparser. We favored pycparser
over PyCParser and cppheaderparser because it has better documentation, seems
to be a more mature project, and it supports the complete C99 specification,
which PyCParser does not.

For unit testing we have selected the Attest library, as it is the simplest of
the ones considered, and therefore easiest to learn.

A challenge with using Python is the fact that not all team members have
sufficient experience with it. Most team members must therefore do some self
study of varying level before we start the first sprint.

\subsection{Configuration Choice}
According to the given requirements, we decided to use YAML for handling
configuration files for our purposes. The choice was made dependently on used
programming language decisions. Because the latest version of Python
specification (currently 3.2) was chosen, the range of possible configuration
file solutions was reduced. Therefore, although ASN.1 and ConfigObj are very
suitable for our task, they were eliminated (those parsers are available only
up to version 2.7), which left us 2 main possibilities: YAML and configparser.
Configparser turned out to be not sufficient for us, because mainly it lacks
lists. These we need for description of hiearchical structures of the C
headers. YAML has only two minor disadvantages we should be aware of. First,
there is no type validation mechanism, so we will have to create our validation
manually. Second, it is an external library. We find this drawback minor for
now but it can turn up to be a problem in the future. Except for these issues,
YAML (toghether with PyYAML) seems to have a good potential for creating
flexible configuration support for our solution.

Some of the possible advantages and drawbacks of solutions taken into
consideration are depicted in table below:
\begin{table}[H] \footnotesize \center
\caption{Configuration overview}
\noindent\makebox[\textwidth]{%
\begin{tabularx}{1.2\textwidth}{X X X X X}
	\toprule
	& YAML & configparser & ConfigObj & ASN.1 \\
	\midrule
	Advantages &
	+Simplicity\newline +Flexibility &
	+Easy to use &
	+Easy to use\newline +Flexibility\newline +Nesting\newline +Type validation &
	+Customer wants it \\
	\addlinespace
	Drawbacks &
	-External library\newline -No type validation &
	-Lacks nesting\newline -Lacks lists\newline -No type validation &
	-External library\newline -Lacks lists &
	-External library\newline -Too generic \\
	\addlinespace
	Current version & 3.10 & 3.2 & 4.7.2 & 0.0.13 \\
	Supported python & 2.7 and 3.3 & 2.7 and 3.3 & 2.7 & 2.7 \\
	License & MIT & PSF L & BSD-new & BSD \\
	\bottomrule
\end{tabularx}}
\end{table}

\subsection{Integrated Development Environment Choice}
PyCharm is by far the best IDE evaluated in terms of functionality, and it is
the one that mirrors Eclipse the most, which is an advantage, since most team
members are best acquainted with Eclipse. It will be the recommended IDE for
this project, given that we can acquire classroom licenses.

On the other hand, there is no real reason to dictate the use of IDE, since
what determines the productivity of a team member is more how well you know
the specific tool you are using. It will therefore be up to each team member
to choose what IDE/text editor they want to use.

\section{Framework}
%------------------
\label{sec:pre:framework}
Even will descripe the framework we have decided to create?

\subsection{Versions}
The versions of the different tools and libraries we are using.
\begin{description}
	\item[Python] CPython 3.2.2, the latest standard python implementation
		which means we won’t support other python implementation like PyPy,
		jython or ironpython in the near future.
	\item[PLY] 3.4, latest stable verison.
	\item[pycparser] 2.05-dev, latest development version (stable don't
		support \_Bool type)
	\item[PyYAML] 3.10, latest stable version
	\item[Attest] 0.6-dev, latest development version (stable don’t
		support python3.2)
	\item[WireShark] 1.7.0-SVN, latest nightly build, bleeding edge
\end{description}

\section{IP Rights \& License}
%----------------------------
\label{sec:pre:license}
The customer has explained that they do not intend to distribute our utility,
and that we are free to license it as open source if we want to, under
whichever license we feel is most suited. They suggested GNU GPL as Wireshark
is released under it.

The libraries and tools we depend upon and interact with has their own licenses
which we need to consider when we select a license for our utility.
\begin{description}
	\item[Wireshark] GNU GPL
	\item[PLY] BSD-new
	\item[pycparser] BSD-new
	\item[Python-ASN1] - MIT
	\item[pyYAML] - MIT
	\item[ConfigObj] - BSD
\end{description}

\noindent Some of the requirements for our utility might require us to modify
the C preprocessor in PLY and the pycparser library, which made us consider
the BSD the most suited license for us. Since it also gives us the option to
later move to a more restrictive license like GPL, we decided to use it.

