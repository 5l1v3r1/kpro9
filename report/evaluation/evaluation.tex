%===========================
\chapter{Project Evaluation}
%===========================
This chapter gives an evaluation of the project and the course, TDT4290 Customer Driven Project. The main focus is to describe the how the team evolved and how work was done during the project.   

%----------------------------------
\section{Team Dynamics}
%----------------------------------
The forming and development of the team are described in this section.
%----------------------------------
\subsection{Goals And Team Building}
%----------------------------------
The project started out with randomly assigned student groups of six to seven people. This was done intentionally to learn the students to work in a realistic setting. Our group consisted of six Norwegian students and one Czech student. 

As one of the team members was a foreign student, all the internal team communication had to be done in english. In addition our advisors were english speaking, so the whole project was done in english. This did not occur as a problem, because all the team members both spoke and wrote fluent english.

At the first meeting we decided to state our personal goals for the project. This resulted in the following list:
\begin{itemize}
\item Improve programming skills.
\item Learn to develop software efficiently.
\item Learn to work in a realistic environment.
\item Fulfill the needs of the customer. 
\end{itemize}


\subsection{Team evolution}
%---------------------------------
The first weeks of the project the team were in a good, but also uncertain stage. The team members did not know each others boundaries and tried to not end up in an argument. This resulted in a series of matters, including low work effort and bad commitment. This all changed when the team members got more familiar with each other.   

\section{Risk handling}
%----------------------------------
Some of the risks predicted in the planning phase occured to a certain degree during the project.
This section will discuss those risks and how they were handled.

\paragraph{R4. Illness/Absence}
In general, not many team members were absent for longer periods.
One team member was away in northern Norway for a week on vacation, and another was sick for a week and a half. This caused some delay on a few tasks, as the absentees had to get up to date on the state of the project, but it did not majorly hinder the progress of the team. The consequence of this risk was also diminished by the fact that it occured early in the project, and that the team members who were absent did not work on critical tasks at the time.

\paragraph{R6. Conflicts within team}
In the start of the project, the team members were divided on which tools to use, and on which programming language to use. Because of this, the team had to spend some time discussing back and forth. After some constructive discussions the team was able to come to a mutual decision.

Also, some team members felt that it was not realistic that the course should demand 25 hours from each member. As the project progressed, it quickly became apparent that this number of hours were needed to complete the project in a satisfying manner. This led to an overall increase in work effort in the team.

\paragraph{R8. Miscommunication within team}
During sprint 3, the ones responsible for creating test data and the ones writing the test cases did not clearly communicate with each other. This led to having to make small changes in some of the test cases to be able to use the test data.

The test responsible wrote test cases for functionality that the programming team had not thought about. For example, checking that you are not allowed two platforms with the same platform ID.
When the problem was detected, all essential functionality was added.

\paragraph{R10. Lack of experience with Scrum}
As the team had no previous experience with Scrum, the project got off to a slow start.
After evaluating the first sprint, it quickly became apparent that the process was far from perfect.
The second sprint was an improvement of the first, and the planning meeting was longer and more detailed, but in our opinion it was still not good enough. We felt that we did not adhere to proper Scrum, and that we did not properly explain the different tasks in the backlog. In the last two sprints, we felt that we had achieved a better understanding, and this really showed in the process. A more detailed discussion can be found in the Scrum section.

\paragraph{R11. Requirements added or modified late in the project}
At the customer meeting on the first day of sprint 4, the customer suggested several new requirements that they would like is to implement. Some requirements were also modified, as we had not implemented them exactly the way they wanted us to.

As we had to focus on tweaking some functionalities to work on their code, and also had to spend time on writing the report, we had to tell the customer that we would probably not be able to finish all the new requirements. This is because implementing a requirement would also require testing, user documentation and additional report work, which is something we could not afford to allot time for.


%--------------------------------
\section{The Process}
%under construction!
%--------------------------------
\subsection{Scrum}
%--------------------------------

%----------------
\section{Lectures}
%----------------
%short section

%----------------
\section{Advisors}
%----------------
%short section

%----------------
\section{Customer}
%----------------
%short section

%----------------
\section{Summary}
%----------------

